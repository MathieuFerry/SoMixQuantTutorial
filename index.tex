% Options for packages loaded elsewhere
\PassOptionsToPackage{unicode}{hyperref}
\PassOptionsToPackage{hyphens}{url}
\PassOptionsToPackage{dvipsnames,svgnames,x11names}{xcolor}
%
\documentclass[
  letterpaper,
  DIV=11,
  numbers=noendperiod]{scrreprt}

\usepackage{amsmath,amssymb}
\usepackage{iftex}
\ifPDFTeX
  \usepackage[T1]{fontenc}
  \usepackage[utf8]{inputenc}
  \usepackage{textcomp} % provide euro and other symbols
\else % if luatex or xetex
  \usepackage{unicode-math}
  \defaultfontfeatures{Scale=MatchLowercase}
  \defaultfontfeatures[\rmfamily]{Ligatures=TeX,Scale=1}
\fi
\usepackage{lmodern}
\ifPDFTeX\else  
    % xetex/luatex font selection
\fi
% Use upquote if available, for straight quotes in verbatim environments
\IfFileExists{upquote.sty}{\usepackage{upquote}}{}
\IfFileExists{microtype.sty}{% use microtype if available
  \usepackage[]{microtype}
  \UseMicrotypeSet[protrusion]{basicmath} % disable protrusion for tt fonts
}{}
\makeatletter
\@ifundefined{KOMAClassName}{% if non-KOMA class
  \IfFileExists{parskip.sty}{%
    \usepackage{parskip}
  }{% else
    \setlength{\parindent}{0pt}
    \setlength{\parskip}{6pt plus 2pt minus 1pt}}
}{% if KOMA class
  \KOMAoptions{parskip=half}}
\makeatother
\usepackage{xcolor}
\setlength{\emergencystretch}{3em} % prevent overfull lines
\setcounter{secnumdepth}{5}
% Make \paragraph and \subparagraph free-standing
\makeatletter
\ifx\paragraph\undefined\else
  \let\oldparagraph\paragraph
  \renewcommand{\paragraph}{
    \@ifstar
      \xxxParagraphStar
      \xxxParagraphNoStar
  }
  \newcommand{\xxxParagraphStar}[1]{\oldparagraph*{#1}\mbox{}}
  \newcommand{\xxxParagraphNoStar}[1]{\oldparagraph{#1}\mbox{}}
\fi
\ifx\subparagraph\undefined\else
  \let\oldsubparagraph\subparagraph
  \renewcommand{\subparagraph}{
    \@ifstar
      \xxxSubParagraphStar
      \xxxSubParagraphNoStar
  }
  \newcommand{\xxxSubParagraphStar}[1]{\oldsubparagraph*{#1}\mbox{}}
  \newcommand{\xxxSubParagraphNoStar}[1]{\oldsubparagraph{#1}\mbox{}}
\fi
\makeatother

\usepackage{color}
\usepackage{fancyvrb}
\newcommand{\VerbBar}{|}
\newcommand{\VERB}{\Verb[commandchars=\\\{\}]}
\DefineVerbatimEnvironment{Highlighting}{Verbatim}{commandchars=\\\{\}}
% Add ',fontsize=\small' for more characters per line
\usepackage{framed}
\definecolor{shadecolor}{RGB}{241,243,245}
\newenvironment{Shaded}{\begin{snugshade}}{\end{snugshade}}
\newcommand{\AlertTok}[1]{\textcolor[rgb]{0.68,0.00,0.00}{#1}}
\newcommand{\AnnotationTok}[1]{\textcolor[rgb]{0.37,0.37,0.37}{#1}}
\newcommand{\AttributeTok}[1]{\textcolor[rgb]{0.40,0.45,0.13}{#1}}
\newcommand{\BaseNTok}[1]{\textcolor[rgb]{0.68,0.00,0.00}{#1}}
\newcommand{\BuiltInTok}[1]{\textcolor[rgb]{0.00,0.23,0.31}{#1}}
\newcommand{\CharTok}[1]{\textcolor[rgb]{0.13,0.47,0.30}{#1}}
\newcommand{\CommentTok}[1]{\textcolor[rgb]{0.37,0.37,0.37}{#1}}
\newcommand{\CommentVarTok}[1]{\textcolor[rgb]{0.37,0.37,0.37}{\textit{#1}}}
\newcommand{\ConstantTok}[1]{\textcolor[rgb]{0.56,0.35,0.01}{#1}}
\newcommand{\ControlFlowTok}[1]{\textcolor[rgb]{0.00,0.23,0.31}{\textbf{#1}}}
\newcommand{\DataTypeTok}[1]{\textcolor[rgb]{0.68,0.00,0.00}{#1}}
\newcommand{\DecValTok}[1]{\textcolor[rgb]{0.68,0.00,0.00}{#1}}
\newcommand{\DocumentationTok}[1]{\textcolor[rgb]{0.37,0.37,0.37}{\textit{#1}}}
\newcommand{\ErrorTok}[1]{\textcolor[rgb]{0.68,0.00,0.00}{#1}}
\newcommand{\ExtensionTok}[1]{\textcolor[rgb]{0.00,0.23,0.31}{#1}}
\newcommand{\FloatTok}[1]{\textcolor[rgb]{0.68,0.00,0.00}{#1}}
\newcommand{\FunctionTok}[1]{\textcolor[rgb]{0.28,0.35,0.67}{#1}}
\newcommand{\ImportTok}[1]{\textcolor[rgb]{0.00,0.46,0.62}{#1}}
\newcommand{\InformationTok}[1]{\textcolor[rgb]{0.37,0.37,0.37}{#1}}
\newcommand{\KeywordTok}[1]{\textcolor[rgb]{0.00,0.23,0.31}{\textbf{#1}}}
\newcommand{\NormalTok}[1]{\textcolor[rgb]{0.00,0.23,0.31}{#1}}
\newcommand{\OperatorTok}[1]{\textcolor[rgb]{0.37,0.37,0.37}{#1}}
\newcommand{\OtherTok}[1]{\textcolor[rgb]{0.00,0.23,0.31}{#1}}
\newcommand{\PreprocessorTok}[1]{\textcolor[rgb]{0.68,0.00,0.00}{#1}}
\newcommand{\RegionMarkerTok}[1]{\textcolor[rgb]{0.00,0.23,0.31}{#1}}
\newcommand{\SpecialCharTok}[1]{\textcolor[rgb]{0.37,0.37,0.37}{#1}}
\newcommand{\SpecialStringTok}[1]{\textcolor[rgb]{0.13,0.47,0.30}{#1}}
\newcommand{\StringTok}[1]{\textcolor[rgb]{0.13,0.47,0.30}{#1}}
\newcommand{\VariableTok}[1]{\textcolor[rgb]{0.07,0.07,0.07}{#1}}
\newcommand{\VerbatimStringTok}[1]{\textcolor[rgb]{0.13,0.47,0.30}{#1}}
\newcommand{\WarningTok}[1]{\textcolor[rgb]{0.37,0.37,0.37}{\textit{#1}}}

\providecommand{\tightlist}{%
  \setlength{\itemsep}{0pt}\setlength{\parskip}{0pt}}\usepackage{longtable,booktabs,array}
\usepackage{calc} % for calculating minipage widths
% Correct order of tables after \paragraph or \subparagraph
\usepackage{etoolbox}
\makeatletter
\patchcmd\longtable{\par}{\if@noskipsec\mbox{}\fi\par}{}{}
\makeatother
% Allow footnotes in longtable head/foot
\IfFileExists{footnotehyper.sty}{\usepackage{footnotehyper}}{\usepackage{footnote}}
\makesavenoteenv{longtable}
\usepackage{graphicx}
\makeatletter
\def\maxwidth{\ifdim\Gin@nat@width>\linewidth\linewidth\else\Gin@nat@width\fi}
\def\maxheight{\ifdim\Gin@nat@height>\textheight\textheight\else\Gin@nat@height\fi}
\makeatother
% Scale images if necessary, so that they will not overflow the page
% margins by default, and it is still possible to overwrite the defaults
% using explicit options in \includegraphics[width, height, ...]{}
\setkeys{Gin}{width=\maxwidth,height=\maxheight,keepaspectratio}
% Set default figure placement to htbp
\makeatletter
\def\fps@figure{htbp}
\makeatother

\KOMAoption{captions}{tableheading}
\makeatletter
\@ifpackageloaded{tcolorbox}{}{\usepackage[skins,breakable]{tcolorbox}}
\@ifpackageloaded{fontawesome5}{}{\usepackage{fontawesome5}}
\definecolor{quarto-callout-color}{HTML}{909090}
\definecolor{quarto-callout-note-color}{HTML}{0758E5}
\definecolor{quarto-callout-important-color}{HTML}{CC1914}
\definecolor{quarto-callout-warning-color}{HTML}{EB9113}
\definecolor{quarto-callout-tip-color}{HTML}{00A047}
\definecolor{quarto-callout-caution-color}{HTML}{FC5300}
\definecolor{quarto-callout-color-frame}{HTML}{acacac}
\definecolor{quarto-callout-note-color-frame}{HTML}{4582ec}
\definecolor{quarto-callout-important-color-frame}{HTML}{d9534f}
\definecolor{quarto-callout-warning-color-frame}{HTML}{f0ad4e}
\definecolor{quarto-callout-tip-color-frame}{HTML}{02b875}
\definecolor{quarto-callout-caution-color-frame}{HTML}{fd7e14}
\makeatother
\makeatletter
\@ifpackageloaded{bookmark}{}{\usepackage{bookmark}}
\makeatother
\makeatletter
\@ifpackageloaded{caption}{}{\usepackage{caption}}
\AtBeginDocument{%
\ifdefined\contentsname
  \renewcommand*\contentsname{Table of contents}
\else
  \newcommand\contentsname{Table of contents}
\fi
\ifdefined\listfigurename
  \renewcommand*\listfigurename{List of Figures}
\else
  \newcommand\listfigurename{List of Figures}
\fi
\ifdefined\listtablename
  \renewcommand*\listtablename{List of Tables}
\else
  \newcommand\listtablename{List of Tables}
\fi
\ifdefined\figurename
  \renewcommand*\figurename{Figure}
\else
  \newcommand\figurename{Figure}
\fi
\ifdefined\tablename
  \renewcommand*\tablename{Table}
\else
  \newcommand\tablename{Table}
\fi
}
\@ifpackageloaded{float}{}{\usepackage{float}}
\floatstyle{ruled}
\@ifundefined{c@chapter}{\newfloat{codelisting}{h}{lop}}{\newfloat{codelisting}{h}{lop}[chapter]}
\floatname{codelisting}{Listing}
\newcommand*\listoflistings{\listof{codelisting}{List of Listings}}
\makeatother
\makeatletter
\makeatother
\makeatletter
\@ifpackageloaded{caption}{}{\usepackage{caption}}
\@ifpackageloaded{subcaption}{}{\usepackage{subcaption}}
\makeatother

\ifLuaTeX
  \usepackage{selnolig}  % disable illegal ligatures
\fi
\usepackage{bookmark}

\IfFileExists{xurl.sty}{\usepackage{xurl}}{} % add URL line breaks if available
\urlstyle{same} % disable monospaced font for URLs
\hypersetup{
  pdftitle={Formation R},
  pdfauthor={Louise Caron; Mathieu Ferry},
  colorlinks=true,
  linkcolor={blue},
  filecolor={Maroon},
  citecolor={Blue},
  urlcolor={Blue},
  pdfcreator={LaTeX via pandoc}}


\title{Formation R}
\author{Louise Caron \and Mathieu Ferry}
\date{2026-01-11}

\begin{document}
\maketitle

\renewcommand*\contentsname{Table of contents}
{
\hypersetup{linkcolor=}
\setcounter{tocdepth}{2}
\tableofcontents
}

\bookmarksetup{startatroot}

\chapter*{Preface}\label{preface}
\addcontentsline{toc}{chapter}{Preface}

\markboth{Preface}{Preface}

This short guide summarizes the learning elements proposed for the SoMix
training in December 2025 in Colombo (Sri Lanka).

The various tutorials aim to familiarize participants with the RStudio
console, the R language (and the tidyverse) through a selection of
functions and tools considered most relevant for handling statistical
data in the social sciences.

This guide is a training support for R; it is not a full course in
statistics. It is also not intended to replace the many excellent guides
available on the internet, for example:

\begin{itemize}
\tightlist
\item
  \href{https://r4ds.hadley.nz/}{R for data science}
\end{itemize}

\part{\textbf{Discovering R}}

\chapter{Introduction}\label{introduction}

\section{Install R and RStudio}\label{install-r-and-rstudio}

We will start by installing R according to your computer's
configuration, using the links available on the
\href{https://cran.r-project.org/}{Comprenhensive R Archive Network}
(CRAN).

We will then install RStudio by going to this
\href{https://posit.co/download/rstudio-desktop/}{page}. RStudio is an
``Integrated Development Environment (IDE)'', meaning software that
provides a user-friendly interface for working with the R programming
language. Note that it is possible to use other interface software, but
RStudio is probably the most widely used today.

If, for one reason or another, it is not possible to install R / RStudio
on your computer, you can temporarily (because we do not recommend using
a private server to store sensitive data) use R / RStudio online by
creating an account on the
\href{https://login.posit.cloud/login?redirect=\%2Foauth\%2Fauthorize\%3Fredirect_uri\%3Dhttps\%253A\%252F\%252Fposit.cloud\%252Flogin\%26client_id\%3Dposit-cloud\%26response_type\%3Dcode\%26show_auth\%3D0&product=cloud}{``Posit
cloud''}.

Once R and RStudio are installed, you can directly open RStudio, which
will display a nice console (except that here, I have already created an
R script by clicking on the small white arrow on a green background and
saved it under the name 1Intro.R, which you should also do).

\begin{tcolorbox}[enhanced jigsaw, toprule=.15mm, colbacktitle=quarto-callout-note-color!10!white, opacitybacktitle=0.6, opacityback=0, breakable, colback=white, coltitle=black, rightrule=.15mm, titlerule=0mm, toptitle=1mm, title={Exercise}, colframe=quarto-callout-note-color-frame, leftrule=.75mm, arc=.35mm, bottomtitle=1mm, bottomrule=.15mm, left=2mm]

\begin{itemize}
\item
  Open RStudio and create an empty script.
\item
  Save this script in a folder (for example, ``Formation R''), naming it
  for instance 1Intro.R.
\end{itemize}

\end{tcolorbox}

\section{R, RStudio and the
tidyverse}\label{r-rstudio-and-the-tidyverse}

You can refer to the following
\href{https://michael.hahsler.net/SMU/DS_Workshop_Intro_R/slides/3a_tidyverse.html}{small
presentation} of the tidyverse. Let us recall here that R is a
programming language developed since the 1990s, derived from the S
language, and sharing some similarities with the C language. It is
open-source and free of charge.

Among the many strengths of R, one can note in particular its community
of users, who can also take on the role of developers. R therefore has
numerous extensions in the form of functions, stored in packages (or
libraries), which greatly facilitate its use. The list of packages
available on CRAN (although it is also possible to use packages that are
not on CRAN) is available here:
\url{https://cran.r-project.org/web/packages/}.

With more than 22,000 packages, it's easy to get lost. Throughout this
guide, we suggest using a few packages considered helpful for making
life easier when doing statistics in the social sciences. Again, the
goal is not to be exhaustive, nor to impose anything. One particularity
of R is that it is often possible to do the same thing using different
operations / functions / packages. Everyone eventually finds their own
little ``tricks''!

The tidyverse package actually includes a suite of packages designed to
work together, making data manipulation, recoding, and the production of
graphics easier compared to ``base R''. We will rely on it here for our
statistical work.

\begin{tcolorbox}[enhanced jigsaw, toprule=.15mm, colbacktitle=quarto-callout-note-color!10!white, opacitybacktitle=0.6, opacityback=0, breakable, colback=white, coltitle=black, rightrule=.15mm, titlerule=0mm, toptitle=1mm, title={Exercise}, colframe=quarto-callout-note-color-frame, leftrule=.75mm, arc=.35mm, bottomtitle=1mm, bottomrule=.15mm, left=2mm]

Install the tidyverse package, then load it into the R session using the
two lines of code below. You can copy-paste them into your script and
run them (see the next section if you don't know how to run lines of
code).

\end{tcolorbox}

\begin{Shaded}
\begin{Highlighting}[]
\FunctionTok{install.packages}\NormalTok{(}\StringTok{"tidyverse"}\NormalTok{)}
\FunctionTok{library}\NormalTok{(tidyverse)}
\end{Highlighting}
\end{Shaded}

\begin{tcolorbox}[enhanced jigsaw, toprule=.15mm, colbacktitle=quarto-callout-note-color!10!white, opacitybacktitle=0.6, opacityback=0, breakable, colback=white, coltitle=black, rightrule=.15mm, titlerule=0mm, toptitle=1mm, title=\textcolor{quarto-callout-note-color}{\faInfo}\hspace{0.5em}{Note}, colframe=quarto-callout-note-color-frame, leftrule=.75mm, arc=.35mm, bottomtitle=1mm, bottomrule=.15mm, left=2mm]

You only need to install packages once; however, you must load the
package each time you open a new R session. So, if you close RStudio and
then reopen it, there is no need to run
\texttt{install.packages("tidyverse")} again --- but to use the
functions from this package, you still need to call tidyverse by running
\texttt{library(tidyverse)}.

Calling the relevant packages is generally done systematically at the
beginning of the script.

\end{tcolorbox}

\section{RStudio as a working tool}\label{rstudio-as-a-working-tool}

By default, the console is displayed in the form of four panels:

\begin{itemize}
\item
  The first panel in the top left corresponds to the R script file,
  which for now is empty (or almost empty, if you have already
  copy-pasted the lines of code to install and load tidyverse). This is
  where we will write the command lines that we will run, either by
  clicking Run, or by using the keyboard shortcut Cmd + Enter (or
  Control + Enter).
\item
  The second panel in the bottom left corresponds to the console. For
  now, it simply tells us that it has loaded R in the most up-to-date
  version found on my computer (4.4.3). The console will display the
  commands that we have run and the messages returned by R when
  executing them.
\item
  The third panel in the top right contains several tabs. The most
  important one is the Environment tab, which shows all the objects
  created in the current session (data frames, lists, vectors, etc.).
\item
  The fourth panel in the bottom right also contains several tabs,
  including Files --- which allows you to browse your computer's
  directory structure --- Plots, where our beautiful graphs will be
  displayed, Packages, which lists the packages (libraries) installed on
  our computer and indicates which ones are loaded in the current
  session (they are checked), Help, which displays information about a
  function when you type \texttt{?function\_name} in the console, and
  Viewer, where our nice tables will appear.
\end{itemize}

\begin{tcolorbox}[enhanced jigsaw, toprule=.15mm, colbacktitle=quarto-callout-note-color!10!white, opacitybacktitle=0.6, opacityback=0, breakable, colback=white, coltitle=black, rightrule=.15mm, titlerule=0mm, toptitle=1mm, title={Exercise}, colframe=quarto-callout-note-color-frame, leftrule=.75mm, arc=.35mm, bottomtitle=1mm, bottomrule=.15mm, left=2mm]

Check that the tidyverse package is correctly installed and loaded in
the R session.

\end{tcolorbox}

\begin{figure}[H]

{\centering \includegraphics{images/clipboard-343351687.png}

}

\caption{The RStudio console}

\end{figure}%

\begin{tcolorbox}[enhanced jigsaw, toprule=.15mm, colbacktitle=quarto-callout-caution-color!10!white, opacitybacktitle=0.6, opacityback=0, breakable, colback=white, coltitle=black, rightrule=.15mm, titlerule=0mm, toptitle=1mm, title=\textcolor{quarto-callout-caution-color}{\faFire}\hspace{0.5em}{Caution}, colframe=quarto-callout-caution-color-frame, leftrule=.75mm, arc=.35mm, bottomtitle=1mm, bottomrule=.15mm, left=2mm]

\begin{itemize}
\item
  When quitting RStudio, a window may appear asking whether you want to
  ``Save the workspace image\ldots{}''. And in fact, it is not a good
  idea to click Save!
\item
  Because this saves the entire workspace (the session), which will be
  automatically reopened when restarting RStudio, and you may end up
  with objects that are no longer relevant for a later session (and that
  may even conflict with one another).
\item
  If you did not click Save, reopening RStudio resets the R session. To
  make sure this is really the case (clear the global environment,
  unload packages, reset memory, etc.), you can type the following in
  the console:
\end{itemize}

\begin{Shaded}
\begin{Highlighting}[]
\FunctionTok{.rs.restartR}\NormalTok{()}
\end{Highlighting}
\end{Shaded}

\end{tcolorbox}

\section{Loading data into R}\label{loading-data-into-r}

To do statistics, we need data. Here, we are going to use excerpts of
the
\href{https://www.statistics.gov.lk/IncomeAndExpenditure/StaticalInformation}{Household
Income and Expenditure Survey 2019} collected by the Sri Lankan
Department of Census and Statistics. For legality matters, you will need
to create an account on the
\href{https://nada.statistics.gov.lk/index.php/home}{Lanka data
platform} and download the microdata files from the online platform.
Notice that you are able to download only 25\% of the total survey
sample.

For this workshop, we have additionally prepared three files to make
your life easier to manipulate the survey, which you can access here
\href{https://forms.gle/tkBxwuPQtSvmfT9o7}{Data repository}. You will
need to provide a password (request password by email!). Download the
files from this online folder. The folder contains a PDF of the
questionnaire and a zipped folder which you can unzip on your computer.
In the zipped folder, you will find three ``.rds'' files (R statistical
files): - \texttt{edu.rds} contains data from Section 2 of the survey,
i.e.~on school education for household members aged 5 to 19 -
\texttt{health.rds} contains data from Section 3A of the survey, i.e.~on
health treatment and chronic illness - \texttt{environment.rds} contains
data from Section 7 \& 8, i.e.~on access to primary facilities and
housing information

Notice that the unit of analysis differs in the three different files.
For \texttt{edu.rds}, the unit of analysis (a ``row'' in the data file)
are all household individuals aged 5 to 19 (N=4,552). For
\texttt{health.rds} and \texttt{environment.rds}, the

In addition the variables (the different ``columns'' of the files) of
these sections, we have already added a number of socio-demographic
variables in each of these files to ease statistical manipulations,
including: - \texttt{hhid}: a unique household identifier -
\texttt{indid}: a unique individual identifier (only in
\texttt{edu.rds}) - \texttt{edu\_attained}: this is a recoded version of
individual educational attainment - \texttt{edu\_father}: father's
educational attainment (only in \texttt{edu.rds}) -
\texttt{edu\_mother}: mother's educational attainment (only in
\texttt{edu.rds}) - \texttt{edu\_parent}: combined parents' educational
attainment, i.e.~maximum educational attainment between both parents
(only in \texttt{edu.rds}) - \texttt{age}: individual's age (only in
\texttt{edu.rds}) - \texttt{birth\_year}: individual's age (only in
\texttt{edu.rds}) - \texttt{sex}: individual's sex (only in
\texttt{edu.rds}) - \texttt{age\_father}: father's age (only in
\texttt{edu.rds}) - \texttt{age\_mother}: mother's age (only in
\texttt{edu.rds}) - \texttt{province}: province of residence -
\texttt{ethnicity}: household head's ethnicity - \texttt{religion}:
household head's religion - \texttt{hhwealth}: a continuous variable to
proxy households' material well-being (an index based on durable goods
ownership) - \texttt{hhwealthcat}: a categorical variable to proxy
households' material well-being (an index based on durable goods
ownership) - \texttt{finalweight\_25per}: a ``survey weight'' variable
to use in all statistical treatments to get correct estimates!

Let's load \texttt{edu.rds} into RStudio. Generally, there are three
options to import data into R: - In the bottom-right panel, you can
browse your folder structure to find the appropriate file, and click on
it. - You can click Import Dataset in the top-right panel, then click
Browse and navigate through Finder or File Explorer. - You can directly
use lines of code written in your script to load your file.

The third solution is preferable, in the interest of making your code
reproducible. But the first or the second have the advantage of being
easier when you are not familiar with coding, especially since these
``point-and-click'' solutions also provide the corresponding lines of
code, which you can copy into your script for your next working session!

\begin{tcolorbox}[enhanced jigsaw, toprule=.15mm, colbacktitle=quarto-callout-note-color!10!white, opacitybacktitle=0.6, opacityback=0, breakable, colback=white, coltitle=black, rightrule=.15mm, titlerule=0mm, toptitle=1mm, title={Exercise}, colframe=quarto-callout-note-color-frame, leftrule=.75mm, arc=.35mm, bottomtitle=1mm, bottomrule=.15mm, left=2mm]

Load \texttt{edu.rds} using the first solution. Then, use the code below
to directly use the second solution. Note that the line
\texttt{setwd("/home/groups/3genquanti/SoMix/HIES\ for\ workshop")}
indicates where you have stored your files locally on your computer. You
can retrieve the file path of your files in Finder or File Explorer by
right-clicking on the data file, then selecting Properties or Get Info.

\end{tcolorbox}

\begin{Shaded}
\begin{Highlighting}[]
\FunctionTok{setwd}\NormalTok{(}\StringTok{"/home/groups/3genquanti/SoMix/HIES for workshop"}\NormalTok{)}
\NormalTok{edu}\OtherTok{\textless{}{-}}\FunctionTok{readRDS}\NormalTok{(}\StringTok{"edu.rds"}\NormalTok{)}
\end{Highlighting}
\end{Shaded}

\section{Describing your dataset}\label{describing-your-dataset}

If everything worked properly, the dataset should now appear in the
Environment (top-right tab).

To see the data file (you can also click on it):

\begin{Shaded}
\begin{Highlighting}[]
\FunctionTok{View}\NormalTok{(edu)}
\end{Highlighting}
\end{Shaded}

Several ways of better describing your dataset can be useful:

\begin{itemize}
\item
  You can type the name of the dataset \texttt{edu} in the console,
  which will display the dataset (not recommended!).
\item
  You can also type \texttt{str(edu)} to understand the structure of the
  dataset.
\item
  You can also type \texttt{summary(edu)} to obtain a summary of the
  quantitative variables in the dataset.
\end{itemize}

\section{A few functions to manipulate
data}\label{a-few-functions-to-manipulate-data}

\subsection{Accessing a variable}\label{accessing-a-variable}

The dataset is therefore composed of 38 variables. To access a variable
--- for example the variable age --- you can type:

\begin{Shaded}
\begin{Highlighting}[]
\NormalTok{edu}\SpecialCharTok{$}\NormalTok{age}
\end{Highlighting}
\end{Shaded}

The \texttt{\$} sign indicates that we are accessing the variable age
contained in the edu dataset. This command returns a vector of the
values of the variable.

In the tidyverse, we call objects using pipes, which come in two
different (and largely equivalent) operators:
\texttt{\%\textgreater{}\%} and \texttt{\textbar{}\textgreater{}}. We
will aim to favor the second one (but they are largely
interchangeable!).

For example, to calculate the mean of age, we would write:

\begin{Shaded}
\begin{Highlighting}[]
\NormalTok{edu }\SpecialCharTok{|\textgreater{}} \FunctionTok{summarise}\NormalTok{(}\FunctionTok{mean}\NormalTok{(age))}
\end{Highlighting}
\end{Shaded}

\texttt{summarise} is the function that allows you to summarise the
dataset according to the function specified. Note that we could also
have written:

\begin{Shaded}
\begin{Highlighting}[]
\NormalTok{edu }\SpecialCharTok{|\textgreater{}} \FunctionTok{summarise}\NormalTok{(}\AttributeTok{Mean\_age=}\FunctionTok{mean}\NormalTok{(age))}
\NormalTok{edu }\SpecialCharTok{|\textgreater{}} \FunctionTok{summarise}\NormalTok{(}\StringTok{\textasciigrave{}}\AttributeTok{Mean age}\StringTok{\textasciigrave{}}\OtherTok{=}\FunctionTok{mean}\NormalTok{(age))}
\end{Highlighting}
\end{Shaded}

\subsection{Transforming a variable}\label{transforming-a-variable}

Let's assume the database does not include age but only birth\_year. In
that case, we could re-create age by creating a variable based on
birth\_year. Here, we use mutate to create/transform variables.

\begin{Shaded}
\begin{Highlighting}[]
\NormalTok{edu }\OtherTok{\textless{}{-}}\NormalTok{ edu }\SpecialCharTok{|\textgreater{}} \FunctionTok{mutate}\NormalTok{(}\AttributeTok{age\_created=}\DecValTok{2019}\SpecialCharTok{{-}}\NormalTok{birth\_year)}
\end{Highlighting}
\end{Shaded}

\subsection{Selecting columns}\label{selecting-columns}

To create a database called \texttt{edu2} where we select only certain
variables, we write:

\begin{Shaded}
\begin{Highlighting}[]
\NormalTok{edu2}\OtherTok{\textless{}{-}}\NormalTok{ edu }\SpecialCharTok{\%\textgreater{}\%} \FunctionTok{select}\NormalTok{(hhid,age,sex,edu\_attained)}
\end{Highlighting}
\end{Shaded}

\texttt{select} allows to select columns of the dataframe.

Similarly, \texttt{select} can be used to remove columns:

\begin{Shaded}
\begin{Highlighting}[]
\NormalTok{edu3 }\OtherTok{\textless{}{-}}\NormalTok{ edu2 }\SpecialCharTok{|\textgreater{}} \FunctionTok{select}\NormalTok{(}\SpecialCharTok{{-}}\FunctionTok{c}\NormalTok{(edu\_attained,sex))}
\end{Highlighting}
\end{Shaded}

\subsection{Selecting rows}\label{selecting-rows}

To create a database from \texttt{edu} where we decide to study only
male respondents:

\begin{Shaded}
\begin{Highlighting}[]
\NormalTok{edum }\OtherTok{\textless{}{-}}\NormalTok{ edu }\SpecialCharTok{|\textgreater{}} \FunctionTok{filter}\NormalTok{(sex}\SpecialCharTok{==}\StringTok{"Male"}\NormalTok{)}
\end{Highlighting}
\end{Shaded}

\texttt{filter} allows you to select the rows of the dataset that
satisfy a given condition. The ampersand operator \texttt{\&} means
``and''. If you want to use the logical ``or'' operator, you write:
\texttt{\textbar{}}.

Let's say we want to study only female respondents with less than
primary attainment:

\begin{Shaded}
\begin{Highlighting}[]
\NormalTok{edufl}\OtherTok{\textless{}{-}}\NormalTok{ edu }\SpecialCharTok{|\textgreater{}} \FunctionTok{filter}\NormalTok{(sex}\SpecialCharTok{==}\StringTok{"Female"} \SpecialCharTok{\&}\NormalTok{ edu\_attainment}\SpecialCharTok{==}\StringTok{"Less than primary"}\NormalTok{)}
\end{Highlighting}
\end{Shaded}

\subsection{Renaming a variable}\label{renaming-a-variable}

Let's rename a variable in \texttt{edufl}:

\begin{Shaded}
\begin{Highlighting}[]
\NormalTok{edufl }\OtherTok{\textless{}{-}}\NormalTok{ edufl }\SpecialCharTok{|\textgreater{}} \FunctionTok{rename}\NormalTok{(}\AttributeTok{ethnic\_category=}\NormalTok{ethnicity)}
\end{Highlighting}
\end{Shaded}

The variable \texttt{ethnicity} has been renamed as
\texttt{ethnic\_category}.

\section{Saving data}\label{saving-data}

To save databases from R (because you have done some data management and
want to save your file), there are several solutions.

\subsection{Native R formats}\label{native-r-formats}

\begin{itemize}
\tightlist
\item
  You can save your data in .RData format. This is the simplest option
  --- no hassle --- and the advantage is that you are sure to preserve
  the format of the variables, the labels (we will come back to this
  later), etc.
\end{itemize}

\begin{Shaded}
\begin{Highlighting}[]
\FunctionTok{save}\NormalTok{(edufl,}\AttributeTok{file=}\StringTok{"edufl.RData"}\NormalTok{)}
\end{Highlighting}
\end{Shaded}

Before using this function, we should not forget to run
\texttt{setwd("path\ to\ the\ folder\ where\ the\ file\ will\ be\ saved")}
to indicate where to save the file (or at least to check that R is
indeed pointing to the desired folder using the \texttt{getwd()}
command).

Note that the RData format can perfectly well be used to save several R
objects in the same file:

\begin{Shaded}
\begin{Highlighting}[]
\FunctionTok{save}\NormalTok{(edufl,edum,}\AttributeTok{file=}\StringTok{"eduv2.RData"}\NormalTok{)}
\end{Highlighting}
\end{Shaded}

To reopen an .RData file, you simply need to write:

\begin{Shaded}
\begin{Highlighting}[]
\FunctionTok{load}\NormalTok{(}\StringTok{"eduv2.RData"}\NormalTok{)}
\end{Highlighting}
\end{Shaded}

The drawback of the .RData format is that, since there may be several
objects in the file, we do not know what they are called and we cannot
directly assign a default object name when opening it in the R console.
This can sometimes be risky (imagine that we have a file eduv2.RData
that contains edufl and edum, but that we already have an object called
edum in our environment that is actually different --- when opening
eduv2.RData, the existing object would be overwritten).

For this reason, datasets can be saved individually using the .rds
format with specific functions (this is how you have been provided with
the workshop data):

\begin{Shaded}
\begin{Highlighting}[]
\FunctionTok{saveRDS}\NormalTok{(edum, }\StringTok{"edum.rds"}\NormalTok{)}
\end{Highlighting}
\end{Shaded}

When opening an .rds file, you explicitly control the name that will be
assigned to it in the R environment (which in this case does not
necessarily need to be \texttt{edum}):

\begin{Shaded}
\begin{Highlighting}[]
\NormalTok{edum2 }\OtherTok{\textless{}{-}} \FunctionTok{readRDS}\NormalTok{(}\StringTok{"edum.rds"}\NormalTok{)}
\end{Highlighting}
\end{Shaded}

\subsection{CSV format}\label{csv-format}

\begin{itemize}
\tightlist
\item
  We sometimes need to export our data in CSV format. In that case, we
  use functions from the readr package, such as \texttt{write\_csv}
  (which uses commas as column separators and periods as decimal
  markers).
\end{itemize}

The CSV format with comma separators is the international standard for
CSV files, and it is easily formatted when Excel on one's computer is
configured in English:

\begin{Shaded}
\begin{Highlighting}[]
\FunctionTok{write\_csv}\NormalTok{(edu, }\StringTok{"edu.csv"}\NormalTok{)}
\end{Highlighting}
\end{Shaded}

\subsection{Other formats}\label{other-formats}

\begin{itemize}
\tightlist
\item
  If you want to save your file directly as an Excel file, this is also
  possible. At least two packages allow this: writexl and openxlsx. If
  we install openxlsx (\texttt{install.packages("openxlsx")}):
\end{itemize}

\begin{Shaded}
\begin{Highlighting}[]
\FunctionTok{library}\NormalTok{(openxlsx)}
\FunctionTok{write.xlsx}\NormalTok{(edu, }\StringTok{"edu.xlsx"}\NormalTok{)}
\end{Highlighting}
\end{Shaded}

If you want to read an Excel file, you can use read.xlsx from the same
package. Here, the argument sheet = NULL indicates that we are reading
all the sheets of the Excel file if there are several:

\begin{Shaded}
\begin{Highlighting}[]
\FunctionTok{read.xlsx}\NormalTok{(}\StringTok{"Salaires.xlsx"}\NormalTok{, }\AttributeTok{sheet =} \ConstantTok{NULL}\NormalTok{)}
\end{Highlighting}
\end{Shaded}

\begin{itemize}
\tightlist
\item
  Sometimes we work with colleagues who use other proprietary paid
  software (what an idea!). In that case, we can use the haven package,
  which allows us to save data in SAS, SPSS or Stata format:
\end{itemize}

\begin{Shaded}
\begin{Highlighting}[]
\FunctionTok{library}\NormalTok{(haven)}
\FunctionTok{write\_dta}\NormalTok{(edu, }\StringTok{"edu.dta"}\NormalTok{)}
\FunctionTok{write\_sas}\NormalTok{(edu, }\StringTok{"edu.sas7bdat"}\NormalTok{)}
\FunctionTok{write\_sav}\NormalTok{(edu, }\StringTok{"edu.sav"}\NormalTok{)}
\end{Highlighting}
\end{Shaded}

\emph{Be careful: with the SAS format, the conversion of value labels is
not as straightforward as with other software.}

\section{Exercise}\label{exercise}

\begin{tcolorbox}[enhanced jigsaw, toprule=.15mm, colbacktitle=quarto-callout-note-color!10!white, opacitybacktitle=0.6, opacityback=0, breakable, colback=white, coltitle=black, rightrule=.15mm, titlerule=0mm, toptitle=1mm, title={Exercise}, colframe=quarto-callout-note-color-frame, leftrule=.75mm, arc=.35mm, bottomtitle=1mm, bottomrule=.15mm, left=2mm]

Save as an .rds file a dataset created from edu with the following
conditions:

\begin{verbatim}
-   It contains only the columns hhid, edu_attained, age, and sex.

-   It contains only individuals who, at the time of the survey, are not in school (r2_school_education %in% c(2, 3)).
\end{verbatim}

\end{tcolorbox}

\part{\textbf{Manipulating and analysing data}}

\chapter{Survey data and univariate
statistics}\label{survey-data-and-univariate-statistics}

This chapter introduces the specifics of performing statistics on survey
data based on representative population samples. We focus on the
univariate description of variables, both numerical (quantitative) and
categorical (qualitative).

We will need four packages that have not yet been loaded (and that you
may not have installed yet).

In the code below, we provide a suggested approach to check whether the
necessary packages for this chapter are installed and to load them into
the environment.

We are using the same dataset as in the previous chapter (\texttt{edu}),
which you can also load into RStudio.

\begin{tcolorbox}[enhanced jigsaw, toprule=.15mm, colbacktitle=quarto-callout-note-color!10!white, opacitybacktitle=0.6, opacityback=0, breakable, colback=white, coltitle=black, rightrule=.15mm, titlerule=0mm, toptitle=1mm, title={Exercise}, colframe=quarto-callout-note-color-frame, leftrule=.75mm, arc=.35mm, bottomtitle=1mm, bottomrule=.15mm, left=2mm]

\begin{itemize}
\item
  Create an empty script.
\item
  Save this script in a folder (for instance, ``WinterWorkshop'') and
  name it for instance ``2Statunis.R''.
\item
  Load the \texttt{edu.rds} database and add the commands below to
  install / load the required packages to use the functions demonstrated
  in this chapter.
\end{itemize}

\begin{Shaded}
\begin{Highlighting}[]
\CommentTok{\# List required packages in a vector called load.lib. }
\NormalTok{load.lib }\OtherTok{\textless{}{-}} \FunctionTok{c}\NormalTok{(}\StringTok{"tidyverse"}\NormalTok{,}\StringTok{"questionr"}\NormalTok{,}\StringTok{"Hmisc"}\NormalTok{,}\StringTok{"esquisse"}\NormalTok{,}\StringTok{"kableExtra"}\NormalTok{) }

\NormalTok{install.lib }\OtherTok{\textless{}{-}}\NormalTok{ load.lib[}\SpecialCharTok{!}\NormalTok{load.lib }\SpecialCharTok{\%in\%} \FunctionTok{installed.packages}\NormalTok{()] }\CommentTok{\# Examine packages which are not yet installed}

\ControlFlowTok{for}\NormalTok{ (lib }\ControlFlowTok{in}\NormalTok{ install.lib) }\FunctionTok{install.packages}\NormalTok{(lib,}\AttributeTok{dependencies=}\ConstantTok{TRUE}\NormalTok{) }\CommentTok{\# Install those if required}

\FunctionTok{sapply}\NormalTok{(load.lib,require,}\AttributeTok{character=}\ConstantTok{TRUE}\NormalTok{) }\CommentTok{\# Load all packages. }

\CommentTok{\#Load data}
\FunctionTok{setwd}\NormalTok{(}\StringTok{"/home/groups/3genquanti/SoMix/HIES for workshop"}\NormalTok{)}
\NormalTok{edu}\OtherTok{\textless{}{-}}\FunctionTok{readRDS}\NormalTok{(}\StringTok{"edu.rds"}\NormalTok{)}
\end{Highlighting}
\end{Shaded}

\end{tcolorbox}

\section{Working with survey data}\label{working-with-survey-data}

The \emph{Household Income and Expenditure Survey (HIES)} data
correspond to a sample of households in Sri Lanka. This sample, thanks
to the household selection procedures included in the survey, is
representative of the entire population of Sri Lanka.

Keep in mind that the selection is based on a complete enumeration of
households and a \emph{random} selection of households included in the
sample (in fact, it's a little more complex, but that's the general
idea).

Why use a sample instead of surveying the entire population?

If the survey were exhaustive, it would be a census, which the DCS
conducts separately. Sampling allows the collection of high-quality,
representative data at lower cost and in less time, and all public and
private statistical institutes use this type of survey method.

Generally, sample data include a \textbf{weight variable}. These weights
ensure the representativeness of the sample for the entire population
(to put it simply, in the most common case, the weights correct biases
in data collection because not all selected respondents were actually
surveyed, and the weight corrects for these biases).

These types of data are very common in the social sciences (and the DCS
conducts several sample-based surveys, such as the Time Use Survey, the
Labor Force Survey, etc.), so it is important to know how to handle
them.

Thus, all statistical operations performed on HIES files must take this
weighting into account. For example, in the previous chapter we
calculated the average age of respondents to the education module. Since
we did not use a weight variable, we simply calculated the \textbf{mean
among the respondents}. If we had taken weighting into account, we would
have \emph{inferred} the \textbf{average age of children between 5 and
19 years old for the entire population}.

In our \emph{HIES} files, the correct weight variable is
\texttt{finalweight\_25per} (this is a specific weight variable to this
sample because we are working with a 25\% random sample of the survey
sample, which is itself a random sample of the entire population).

\section{Describing a numerical
variable}\label{describing-a-numerical-variable}

\subsection{Summarise variables based on a few
statistics}\label{summarise-variables-based-on-a-few-statistics}

In R, we can easily summarize a numerical variable using a few
statistical indicators:

\begin{itemize}
\item
  Minimum
\item
  Mean
\item
  Median
\item
  Maximum
\end{itemize}

The minimum and maximum give an idea of the range of values taken by the
variable, while the mean and median are measures of central tendency.

It is often useful to estimate the median in addition to the mean when
studying a numerical variable. Indeed, the mean is sensitive to
``extreme'' values in a distribution.

The median is the value that separates the lower half from the upper
half of a statistical distribution (50\% of the population has a lower
value, 50\% has a higher value). When the mean and median differ
greatly, it indicates that the distribution of the variable's values is
not evenly spread around the mean.

Typically, when studying income, the mean is often much higher than the
median, because a few individuals have very high incomes, which tends to
increase the mean, while the median is a more stable measure of central
tendency.

To obtain these statistics for the variable \texttt{distance} which
indicates the distance from home to school for children:

\begin{Shaded}
\begin{Highlighting}[]
\NormalTok{edu }\SpecialCharTok{|\textgreater{}}
  \FunctionTok{summarise}\NormalTok{(}
    \AttributeTok{min    =} \FunctionTok{min}\NormalTok{(distance, }\AttributeTok{na.rm =} \ConstantTok{TRUE}\NormalTok{),}
    \AttributeTok{mean   =} \FunctionTok{wtd.mean}\NormalTok{(distance,}\AttributeTok{weights=}\NormalTok{finalweight\_25per,}\AttributeTok{na.rm =} \ConstantTok{TRUE}\NormalTok{),}
    \AttributeTok{median =} \FunctionTok{wtd.quantile}\NormalTok{(distance,}\AttributeTok{probs=}\NormalTok{.}\DecValTok{5}\NormalTok{,}\AttributeTok{weights=}\NormalTok{finalweight\_25per,}\AttributeTok{na.rm =} \ConstantTok{TRUE}\NormalTok{),}
    \AttributeTok{max    =} \FunctionTok{max}\NormalTok{(distance, }\AttributeTok{na.rm =} \ConstantTok{TRUE}\NormalTok{),}
    \AttributeTok{n\_na   =} \FunctionTok{sum}\NormalTok{(}\FunctionTok{is.na}\NormalTok{(distance))}
\NormalTok{  )}
\end{Highlighting}
\end{Shaded}

Notice that we use \texttt{finalweight\_25per} to calculate the mean and
the median, but not to calculate the minimum and maximum. Here, the
minimum and maximum age suggest that some children live very close to
their school, but for at least one child, the school is pretty far from
home!

The mean is higher than the median, indicating that a few unusually high
distance values are stretching the upper tail of the distribution,
resulting in a right-skewed distribution.

We also added the number of \texttt{NA} cases in this variables,
i.e.~the number of missing variables. In fact, here this corresponds to
children who actually do not attend school.

We produced such a nice output that we would like to display it as a
table, but copying and pasting from the console doesn't look very
nice\ldots{}

No worries --- there are plenty of packages in R to create beautiful
table outputs. Here, we propose using
\href{https://cran.r-project.org/web/packages/kableExtra/vignettes/awesome_table_in_html.html}{kableExtra}
:

\begin{Shaded}
\begin{Highlighting}[]
\NormalTok{summary}\OtherTok{\textless{}{-}}\NormalTok{edu }\SpecialCharTok{|\textgreater{}}
  \FunctionTok{summarise}\NormalTok{(}
    \AttributeTok{min    =} \FunctionTok{min}\NormalTok{(distance, }\AttributeTok{na.rm =} \ConstantTok{TRUE}\NormalTok{),}
    \AttributeTok{mean   =} \FunctionTok{wtd.mean}\NormalTok{(distance,}\AttributeTok{weights=}\NormalTok{finalweight\_25per,}\AttributeTok{na.rm =} \ConstantTok{TRUE}\NormalTok{),}
    \AttributeTok{median =} \FunctionTok{wtd.quantile}\NormalTok{(distance,}\AttributeTok{probs=}\NormalTok{.}\DecValTok{5}\NormalTok{,}\AttributeTok{weights=}\NormalTok{finalweight\_25per,}\AttributeTok{na.rm =} \ConstantTok{TRUE}\NormalTok{),}
    \AttributeTok{max    =} \FunctionTok{max}\NormalTok{(distance, }\AttributeTok{na.rm =} \ConstantTok{TRUE}\NormalTok{),}
    \AttributeTok{n\_na   =} \FunctionTok{sum}\NormalTok{(}\FunctionTok{is.na}\NormalTok{(distance))}
\NormalTok{  )}
\NormalTok{summary }\SpecialCharTok{|\textgreater{}} \FunctionTok{kbl}\NormalTok{() }\SpecialCharTok{|\textgreater{}} \FunctionTok{kable\_classic}\NormalTok{(}\AttributeTok{full\_width =}\NormalTok{ F)}

\CommentTok{\#Or to have a lower number of digits for the mean: }
\NormalTok{summary }\SpecialCharTok{|\textgreater{}} \FunctionTok{kbl}\NormalTok{(}\AttributeTok{digits=}\DecValTok{2}\NormalTok{) }\SpecialCharTok{|\textgreater{}} \FunctionTok{kable\_classic}\NormalTok{(}\AttributeTok{full\_width =}\NormalTok{ F)}
\end{Highlighting}
\end{Shaded}

The function \texttt{kbl} transforms the summary object into a table in
LaTeX/HTML format, which is displayed in the Viewer (bottom-right
panel), and the function \texttt{kable\_classic} is one of the default
formatting options I chose.

These tables can be customized endlessly (including adding titles,
captions, etc.). The main advantage is that they can be copied and
pasted cleanly into your report, article, or thesis.

\begin{figure}[H]

{\centering \includegraphics{images/Capture d’écran 2025-11-21 à 18.32.47.png}

}

\caption{Table created with kableExtra and displayed in the Viewer}

\end{figure}%

\subsection{Summarizing a quantitative variable with a
graph}\label{summarizing-a-quantitative-variable-with-a-graph}

All of this looks nice and seems to suggest that \texttt{distance} is a
skewed variable with a distribution stretched to the right. Can we
visualize the distribution?

We can use the esquisse package, which provides a point-and-click
solution for creating plots. You need to run the following line (be a
little patient while the page loads). This avoids having to write the
code yourself to create the plot using the \texttt{ggplot2} package.

\begin{Shaded}
\begin{Highlighting}[]
\FunctionTok{esquisser}\NormalTok{(edu)}
\end{Highlighting}
\end{Shaded}

This should load an interface with the different variables from the
dataset at the top, which you will need to drag into the corresponding
boxes.

\begin{figure}[H]

{\centering \includegraphics{images/Capture d’écran 2025-11-21 à 18.39.30.png}

}

\caption{Esquisse interface}

\end{figure}%

By dragging the distance variable into the X box, a histogram is created
automatically. You can modify this representation, for example, by
choosing a density plot instead. The entire graph is customizable using
the options below.

Finally, you can save your plot and also copy the code that generates
the same plot without having to reopen Esquisse!

\section{Describing a categorical
variable}\label{describing-a-categorical-variable}

Describing a numerical variable is nice, but we often have quite a few
categorical variables in our datasets. How can we describe them?

The function \texttt{freqtable} from the \texttt{questionr} package
allows you to get the weighted counts of a categorical variable, and
\texttt{freq} calculates the corresponding percentages (\%val
corresponds to the percentages of non-NA categories).

Note that it is always possible to remove the part
\texttt{weights=finalweight\_25per} to get unweighted counts and

Here, we are interested in the variable r2\_school\_education, which
corresponds to the respondents' school status.

\begin{Shaded}
\begin{Highlighting}[]
\CommentTok{\#|eval: false}

\NormalTok{edu }\SpecialCharTok{|\textgreater{}} \FunctionTok{freqtable}\NormalTok{(r2\_school\_education,}\AttributeTok{weights=}\NormalTok{finalweight\_25per) }\SpecialCharTok{\%\textgreater{}\%} \FunctionTok{freq}\NormalTok{()}
\NormalTok{edu }\SpecialCharTok{|\textgreater{}} \FunctionTok{freqtable}\NormalTok{(r2\_school\_education) }\SpecialCharTok{\%\textgreater{}\%} \FunctionTok{freq}\NormalTok{()}
\NormalTok{edu }\SpecialCharTok{|\textgreater{}} \FunctionTok{freqtable}\NormalTok{(r2\_school\_education,}\AttributeTok{weights=}\NormalTok{finalweight\_25per) }\SpecialCharTok{\%\textgreater{}\%} \FunctionTok{freq}\NormalTok{() }\SpecialCharTok{\%\textgreater{}\%} \FunctionTok{select}\NormalTok{(}\SpecialCharTok{{-}}\NormalTok{n)}
\end{Highlighting}
\end{Shaded}

\section{Recoding a categorical
variable}\label{recoding-a-categorical-variable}

Note that the categories of r2\_school\_education are here coded as 1,
2, and 3. This is because when getting the from the Lanka datta, we got
CSV files without any data labels so we need to recode them ourselves
based on the questionnaire of the survey.

Luckily, this task can be done very easily by using the \texttt{irec()}
interface (from the \texttt{questionr} package created by Julien
Barnier). Type it in the console and the following window will open to
recode a new variable with explicit levels (in the R lingo, these are
factor levels).

\includegraphics{images/Capture d’écran 2025-11-21 à 18.58.30.png}
\includegraphics{images/Capture d’écran 2025-11-21 à 18.58.37.png}
\includegraphics{images/Capture d’écran 2025-11-21 à 18.58.54.png} The
last panel shows you the R code to recode the variable. When clicking on
Done, this code is sent to the Console. \textbf{We strongly advise you
to copy-paste this code into your script to keep track of all your
recoding steps!}

\begin{tcolorbox}[enhanced jigsaw, toprule=.15mm, colbacktitle=quarto-callout-note-color!10!white, opacitybacktitle=0.6, opacityback=0, breakable, colback=white, coltitle=black, rightrule=.15mm, titlerule=0mm, toptitle=1mm, title={Exercise}, colframe=quarto-callout-note-color-frame, leftrule=.75mm, arc=.35mm, bottomtitle=1mm, bottomrule=.15mm, left=2mm]

With the newly recoded variable of school attendance, recompute the \%
distribution of this variable. Then, open Esquisse and draw a graph of
this variable.

\end{tcolorbox}

\section{Recoding a numerical variable to a categorical
variable}\label{recoding-a-numerical-variable-to-a-categorical-variable}

It can be useful to analyze distance to school not as a numerical
variable but as a categorical variable (for example, what proportion of
students have a school less than 1 km away?).

Recoding a quantitative variable into a qualitative variable is
straightforward. In R, we can use another add-in from the questionr
package: icut:

\begin{Shaded}
\begin{Highlighting}[]
\FunctionTok{icut}\NormalTok{()}
\end{Highlighting}
\end{Shaded}

You can then choose the dataset in which you want to recode a variable,
select the variable to recode, and specify the name of the recoded
variable (by default, the old variable name with ``\_rec'' added). This
add-in relies on R's cut() function.

Different recoding options are offered:

\begin{itemize}
\item
  Manually (if you have identified key values, or know key thresholds
  for a variable, for example a poverty level),
\item
  Using a reclassification algorithm (Jenks, which geographers often
  use, or other algorithms),
\item
  Using quantiles, or
\item
  Using equal-width intervals in the distribution.
\end{itemize}

Here, we choose to recode the variable into quartiles. The newly created
variable corresponds to 4 categories, each roughly representing 25\% of
the unweighted respondents. However, this division is not always exact,
especially when many individuals share the same value --- in this case,
the number of kilometers.

\includegraphics{images/Capture d’écran 2025-11-21 à 19.31.25.png}
\includegraphics{images/Capture d’écran 2025-11-21 à 19.37.34.png}
\includegraphics{images/Capture d’écran 2025-11-21 à 19.37.45.png}

You can then click Done. Note that the variable has not yet been
created, but in the console, the code to create it is provided. You just
need to copy and paste it into your script and run it (you can also
modify it directly yourself if you want):

\begin{Shaded}
\begin{Highlighting}[]
\DocumentationTok{\#\# Cutting edu$distance into edu$distance\_rec}
\NormalTok{edu}\SpecialCharTok{$}\NormalTok{distance\_rec }\OtherTok{\textless{}{-}} \FunctionTok{cut}\NormalTok{(edu}\SpecialCharTok{$}\NormalTok{distance,}
  \AttributeTok{include.lowest =} \ConstantTok{TRUE}\NormalTok{,}
  \AttributeTok{right =} \ConstantTok{FALSE}\NormalTok{,}
  \AttributeTok{dig.lab =} \DecValTok{4}\NormalTok{,}
  \AttributeTok{breaks =} \FunctionTok{c}\NormalTok{(}\DecValTok{0}\NormalTok{, }\DecValTok{1}\NormalTok{, }\DecValTok{2}\NormalTok{, }\DecValTok{4}\NormalTok{, }\DecValTok{84}\NormalTok{)}
\NormalTok{)}
\end{Highlighting}
\end{Shaded}

\begin{tcolorbox}[enhanced jigsaw, toprule=.15mm, colbacktitle=quarto-callout-note-color!10!white, opacitybacktitle=0.6, opacityback=0, breakable, colback=white, coltitle=black, rightrule=.15mm, titlerule=0mm, toptitle=1mm, title={Exercise}, colframe=quarto-callout-note-color-frame, leftrule=.75mm, arc=.35mm, bottomtitle=1mm, bottomrule=.15mm, left=2mm]

Study the distribution of the newly created variable distance\_rec.

Save your dataset as a rds file with the name ``edu\_rec.rds''.

\end{tcolorbox}

\chapter{Statistiques bivariées et
graphiques}\label{statistiques-bivariuxe9es-et-graphiques}

Dans ce chapitre, nous abordons maintenant les statistiques bivariées.
Nous utilisons toujours la base edu.rds et les mêmes packages que dans
le chapitre précédent.

\begin{tcolorbox}[enhanced jigsaw, toprule=.15mm, colbacktitle=quarto-callout-note-color!10!white, opacitybacktitle=0.6, opacityback=0, breakable, colback=white, coltitle=black, rightrule=.15mm, titlerule=0mm, toptitle=1mm, title=\textcolor{quarto-callout-note-color}{\faInfo}\hspace{0.5em}{Exercice}, colframe=quarto-callout-note-color-frame, leftrule=.75mm, arc=.35mm, bottomtitle=1mm, bottomrule=.15mm, left=2mm]

\begin{itemize}
\item
  Create an empty R script.
\item
  Save this script in a folder (for instance, ``WinterWorkshop'') and
  name it for instance ``3Statbis.R''.
\item
  Load the edu database and run the commands below to install/load
  packages.
\end{itemize}

\begin{Shaded}
\begin{Highlighting}[]
\CommentTok{\# On liste les packages dont on a besoin dans un vecteur nommé load.lib. }
\NormalTok{load.lib }\OtherTok{\textless{}{-}} \FunctionTok{c}\NormalTok{(}\StringTok{"tidyverse"}\NormalTok{,}\StringTok{"questionr"}\NormalTok{,}\StringTok{"Hmisc"}\NormalTok{,}\StringTok{"esquisse"}\NormalTok{,}\StringTok{"kableExtra"}\NormalTok{) }
\NormalTok{install.lib }\OtherTok{\textless{}{-}}\NormalTok{ load.lib[}\SpecialCharTok{!}\NormalTok{load.lib }\SpecialCharTok{\%in\%} \FunctionTok{installed.packages}\NormalTok{()] }\CommentTok{\# On regarde les paquets qui ne sont pas installés}
\ControlFlowTok{for}\NormalTok{ (lib }\ControlFlowTok{in}\NormalTok{ install.lib) }\FunctionTok{install.packages}\NormalTok{(lib,}\AttributeTok{dependencies=}\ConstantTok{TRUE}\NormalTok{) }\CommentTok{\# On installe ceux{-}ci}
\FunctionTok{sapply}\NormalTok{(load.lib,require,}\AttributeTok{character=}\ConstantTok{TRUE}\NormalTok{) }\CommentTok{\# Et on charge tous les paquets }



\FunctionTok{setwd}\NormalTok{(}\StringTok{"/home/groups/3genquanti/SoMix/HIES for workshop"}\NormalTok{)}
\NormalTok{edu}\OtherTok{\textless{}{-}}\FunctionTok{readRDS}\NormalTok{(}\StringTok{"edu.rds"}\NormalTok{)}
\end{Highlighting}
\end{Shaded}

\end{tcolorbox}

\section{The association between a categorical variable and a numerical
variable}\label{the-association-between-a-categorical-variable-and-a-numerical-variable}

In the chapter on univariate statistics, we discussed several indicators
of the distribution of a numerical variable (minimum, maximum, mean,
median). Thanks to the lines of code already set up, it is very easy to
calculate these statistics for different categories of another
categorical variable.

You simply need to add another function to compute these statistics for
each category of a variable, namely \texttt{group\_by}:

\begin{Shaded}
\begin{Highlighting}[]
\NormalTok{edu }\SpecialCharTok{|\textgreater{}}
  \FunctionTok{group\_by}\NormalTok{(province) }\SpecialCharTok{|\textgreater{}}
  \FunctionTok{summarise}\NormalTok{(}
    \AttributeTok{min    =} \FunctionTok{min}\NormalTok{(distance, }\AttributeTok{na.rm =} \ConstantTok{TRUE}\NormalTok{),}
    \AttributeTok{mean   =} \FunctionTok{wtd.mean}\NormalTok{(distance,}\AttributeTok{weights=}\NormalTok{finalweight\_25per,}\AttributeTok{na.rm =} \ConstantTok{TRUE}\NormalTok{),}
    \AttributeTok{median =} \FunctionTok{wtd.quantile}\NormalTok{(distance,}\AttributeTok{probs=}\NormalTok{.}\DecValTok{5}\NormalTok{,}\AttributeTok{weights=}\NormalTok{finalweight\_25per,}\AttributeTok{na.rm =} \ConstantTok{TRUE}\NormalTok{),}
    \AttributeTok{max    =} \FunctionTok{max}\NormalTok{(distance, }\AttributeTok{na.rm =} \ConstantTok{TRUE}\NormalTok{),}
    \AttributeTok{n\_na   =} \FunctionTok{sum}\NormalTok{(}\FunctionTok{is.na}\NormalTok{(distance))}
\NormalTok{  )}
\end{Highlighting}
\end{Shaded}

Let's study the distance to school according to the place of residence:
urban, rural, or estate. This corresponds to the variable sector, with
categories 1, 2, and 3. We can recode these categories using the icut()
interface, or directly with the following code:

\begin{Shaded}
\begin{Highlighting}[]
\NormalTok{edu}\SpecialCharTok{$}\NormalTok{sector\_rec }\OtherTok{\textless{}{-}}\NormalTok{ edu}\SpecialCharTok{$}\NormalTok{sector  }\SpecialCharTok{|\textgreater{}}
  \FunctionTok{as.character}\NormalTok{() }\SpecialCharTok{|\textgreater{}}
  \FunctionTok{fct\_recode}\NormalTok{(}
    \StringTok{"Urban"} \OtherTok{=} \StringTok{"1"}\NormalTok{,}
    \StringTok{"Rural"} \OtherTok{=} \StringTok{"2"}\NormalTok{,}
    \StringTok{"Estate"} \OtherTok{=} \StringTok{"3"}
\NormalTok{  )}

\NormalTok{edu }\SpecialCharTok{|\textgreater{}}
  \FunctionTok{group\_by}\NormalTok{(sector\_rec) }\SpecialCharTok{|\textgreater{}}
  \FunctionTok{summarise}\NormalTok{(}
    \AttributeTok{min    =} \FunctionTok{min}\NormalTok{(distance, }\AttributeTok{na.rm =} \ConstantTok{TRUE}\NormalTok{),}
    \AttributeTok{mean   =} \FunctionTok{wtd.mean}\NormalTok{(distance,}\AttributeTok{weights=}\NormalTok{finalweight\_25per,}\AttributeTok{na.rm =} \ConstantTok{TRUE}\NormalTok{),}
    \AttributeTok{median =} \FunctionTok{wtd.quantile}\NormalTok{(distance,}\AttributeTok{probs=}\NormalTok{.}\DecValTok{5}\NormalTok{,}\AttributeTok{weights=}\NormalTok{finalweight\_25per,}\AttributeTok{na.rm =} \ConstantTok{TRUE}\NormalTok{),}
    \AttributeTok{max    =} \FunctionTok{max}\NormalTok{(distance, }\AttributeTok{na.rm =} \ConstantTok{TRUE}\NormalTok{),}
    \AttributeTok{n\_na   =} \FunctionTok{sum}\NormalTok{(}\FunctionTok{is.na}\NormalTok{(distance))}
\NormalTok{  )}
\end{Highlighting}
\end{Shaded}

Nothing prevents us from studying variations in the distance to school
according to several categorical variables. For example, we can compute
the mean and the median with respect to both the province and the place
of residence:

\begin{Shaded}
\begin{Highlighting}[]
\NormalTok{edu }\SpecialCharTok{|\textgreater{}}
  \FunctionTok{group\_by}\NormalTok{(province,sector\_rec) }\SpecialCharTok{|\textgreater{}}
  \FunctionTok{summarise}\NormalTok{(}
    \AttributeTok{mean   =} \FunctionTok{wtd.mean}\NormalTok{(distance,}\AttributeTok{weights=}\NormalTok{finalweight\_25per,}\AttributeTok{na.rm =} \ConstantTok{TRUE}\NormalTok{),}
    \AttributeTok{median =} \FunctionTok{wtd.quantile}\NormalTok{(distance,}\AttributeTok{probs=}\NormalTok{.}\DecValTok{5}\NormalTok{,}\AttributeTok{weights=}\NormalTok{finalweight\_25per,}\AttributeTok{na.rm =} \ConstantTok{TRUE}\NormalTok{)}
\NormalTok{  ) }\SpecialCharTok{|\textgreater{}} 
  \FunctionTok{kbl}\NormalTok{(}\AttributeTok{digits=}\DecValTok{1}\NormalTok{) }\SpecialCharTok{|\textgreater{}} \FunctionTok{kable\_classic}\NormalTok{(}\AttributeTok{full\_width =}\NormalTok{ F)}
\end{Highlighting}
\end{Shaded}

\begin{figure}[H]

{\centering \includegraphics{images/Capture d’écran 2025-11-21 à 21.58.25.png}

}

\caption{Urban-rural gap in distance to school by province}

\end{figure}%

The distance to school tends to be higher in rural areas, and this
difference is mainly driven by extreme values. Indeed, it is primarily
the mean, which is more sensitive to large values, that reflects this
gap, more so than the median.

\begin{Shaded}
\begin{Highlighting}[]
\NormalTok{edu }\SpecialCharTok{|\textgreater{}}
  \FunctionTok{group\_by}\NormalTok{(province,sector\_rec) }\SpecialCharTok{|\textgreater{}}
  \FunctionTok{summarise}\NormalTok{(}
    \AttributeTok{mean   =} \FunctionTok{wtd.mean}\NormalTok{(distance,}\AttributeTok{weights=}\NormalTok{finalweight\_25per,}\AttributeTok{na.rm =} \ConstantTok{TRUE}\NormalTok{) }
\NormalTok{  ) }\SpecialCharTok{|\textgreater{}} 
  \FunctionTok{pivot\_wider}\NormalTok{(}\AttributeTok{names\_from=}\NormalTok{sector\_rec,}\AttributeTok{values\_from=}\NormalTok{mean) }\SpecialCharTok{|\textgreater{}}
  \FunctionTok{mutate}\NormalTok{(}\StringTok{\textasciigrave{}}\AttributeTok{Urban{-}Rural difference}\StringTok{\textasciigrave{}}\OtherTok{=}\NormalTok{Urban}\SpecialCharTok{{-}}\NormalTok{Rural) }\SpecialCharTok{|\textgreater{}}
  \FunctionTok{kbl}\NormalTok{(}\AttributeTok{digits=}\DecValTok{1}\NormalTok{) }\SpecialCharTok{|\textgreater{}} \FunctionTok{kable\_classic}\NormalTok{(}\AttributeTok{full\_width =}\NormalTok{ F)}
\end{Highlighting}
\end{Shaded}

\begin{figure}[H]

{\centering \includegraphics{images/Capture d’écran 2025-11-21 à 21.57.52.png}

}

\caption{Urban-rural gap in distance to school by province}

\end{figure}%

We can clearly see from this last analysis that this pattern holds in
all provinces (except North Western), but the gap varies: it is higher
in the Sabaragamuwa and Southern provinces, where the distance to school
for children in estate areas is particularly pronounced.

\subsection{Résumer par la moyenne et
l'écart-type}\label{ruxe9sumer-par-la-moyenne-et-luxe9cart-type}

It is quite common in academic articles to describe quantitative
variables by reporting the mean (a measure of central tendency) and the
standard deviation (a measure of dispersion, equal to the square root of
the variance).

\begin{Shaded}
\begin{Highlighting}[]
\NormalTok{edu }\SpecialCharTok{|\textgreater{}}
  \FunctionTok{group\_by}\NormalTok{(sector\_rec) }\SpecialCharTok{|\textgreater{}}
  \FunctionTok{summarise}\NormalTok{(}
    \AttributeTok{mean=} \FunctionTok{mean}\NormalTok{(distance,}\AttributeTok{na.rm=}\NormalTok{T),}
    \AttributeTok{sd=} \FunctionTok{sd}\NormalTok{(distance,}\AttributeTok{na.rm=}\NormalTok{T),}
\NormalTok{  ) }\SpecialCharTok{|\textgreater{}}
  \FunctionTok{kbl}\NormalTok{(}\AttributeTok{digits=}\DecValTok{1}\NormalTok{) }\SpecialCharTok{|\textgreater{}} \FunctionTok{kable\_classic}\NormalTok{(}\AttributeTok{full\_width =}\NormalTok{ F)}
\end{Highlighting}
\end{Shaded}

\begin{figure}[H]

{\centering \includegraphics{images/Capture d’écran 2025-11-21 à 22.06.22.png}

}

\caption{Mean and standard deviation of distance to school by residence}

\end{figure}%

Why are these two indicators used? First, we can say from this table
that, on average, the distance to school is higher for children in rural
areas than in urban areas. There is also greater variability in the
distance to school in rural areas (sd stands for standard deviation).

\begin{tcolorbox}[enhanced jigsaw, toprule=.15mm, colbacktitle=quarto-callout-tip-color!10!white, opacitybacktitle=0.6, opacityback=0, breakable, colback=white, coltitle=black, rightrule=.15mm, titlerule=0mm, toptitle=1mm, title=\textcolor{quarto-callout-tip-color}{\faLightbulb}\hspace{0.5em}{Tip}, colframe=quarto-callout-tip-color-frame, leftrule=.75mm, arc=.35mm, bottomtitle=1mm, bottomrule=.15mm, left=2mm]

Should we summarize a variable by its mean or its median?

The first option is very common (especially in English-language
journals) but is not necessarily the most appropriate\ldots{}

When working with skewed (i.e., non-normal) distributions:

\begin{itemize}
\tightlist
\item
  The mean is sensitive to extreme values in the distribution.
\item
  It may be best to have a look at both and plot the distributions!
\end{itemize}

Note: in fact, the standard deviation is then also not really a reliable
indicator of dispersion describing the distribution. One may then prefer
to calculate quantiles (e.g.~quartiles, following Tukey's 5 number
summary!).

\end{tcolorbox}

\begin{tcolorbox}[enhanced jigsaw, toprule=.15mm, colbacktitle=quarto-callout-note-color!10!white, opacitybacktitle=0.6, opacityback=0, breakable, colback=white, coltitle=black, rightrule=.15mm, titlerule=0mm, toptitle=1mm, title={Exercise}, colframe=quarto-callout-note-color-frame, leftrule=.75mm, arc=.35mm, bottomtitle=1mm, bottomrule=.15mm, left=2mm]

Using esquisser, reproduce the following plot. Try to improve it by
playing with color palettes, font sizes\ldots{}

What is your take away from this plot and from the previous statistics?

\begin{figure}[H]

{\centering \includegraphics{images/esquisse-plot.png}

}

\caption{Density of the distance to school by residence location}

\end{figure}%

\end{tcolorbox}

\section{The association between two categorical variables and the
cross-tabulation}\label{the-association-between-two-categorical-variables-and-the-cross-tabulation}

\subsection{Some conventions for
cross-tabulations}\label{some-conventions-for-cross-tabulations}

We now come to cross-tabulations to study the associations between two
categorical variables.

First, let's recall a few analysis conventions. It's easy to get
confused when creating cross-tabulations, so keep the following points
in mind:

\begin{itemize}
\item
  What is my \emph{response variable} or \emph{dependent variable},
  i.e., the variable assumed to depend on another factor?
\item
  What is my \emph{independent variable}, i.e., the variable assumed to
  have an effect on the dependent variable?
\end{itemize}

On peut par exemple ici supposer que le fait d'être actuellement
scolarisé (variable r2\_school\_education) dépend du niveau de diplôme
des parents, mais aussi de leur niveau économique, du lieu de résidence,
etc.

\begin{itemize}
\item
  To visualize whether a factor affects a dependent variable, we can
  create a cross-tabulation in which:
\item
  The dependent variable is placed in the columns.
\item
  The independent variable is placed in the rows.
\item
  Row percentages are calculated (we need to normalize the row counts to
  compare them!).
\item
  We then compare the rows within the same column.
\end{itemize}

Of course, this is theory.

There are many cases where we might be more interested in column
percentages, or even total percentages, or we may want to swap rows and
columns. Nevertheless, keeping these simple conventions in mind helps to
stay oriented when navigating statistical analyses.

\subsection{Practical application of the
cross-tabulation}\label{practical-application-of-the-cross-tabulation}

Nous retrouvons les fonctions du package questionr. Pour créer un
tableau croisé des effectifs non pondérés, on pourra écrire :

\begin{Shaded}
\begin{Highlighting}[]
\NormalTok{edu}\SpecialCharTok{$}\NormalTok{r2\_school\_education\_rec }\OtherTok{\textless{}{-}}\NormalTok{ edu}\SpecialCharTok{$}\NormalTok{r2\_school\_education  }\SpecialCharTok{|\textgreater{}}
  \FunctionTok{as.character}\NormalTok{() }\SpecialCharTok{|\textgreater{}}
  \FunctionTok{fct\_recode}\NormalTok{(}
    \StringTok{"Currently attending"} \OtherTok{=} \StringTok{"1"}\NormalTok{,}
    \StringTok{"Never attended"} \OtherTok{=} \StringTok{"2"}\NormalTok{,}
    \StringTok{"Attended school in the past"} \OtherTok{=} \StringTok{"3"}
\NormalTok{  )}

\NormalTok{edu }\SpecialCharTok{|\textgreater{}} \FunctionTok{freqtable}\NormalTok{(edu\_parents,r2\_school\_education\_rec)}
\end{Highlighting}
\end{Shaded}

La variable en ligne (ici, race) est la variable indépendante, c'est la
première à être écrire, tandis que la variable en colonne (ici,
salaire\_quint) est la variable dépendante.

Pour obtenir des pourcentages en ligne, il suffit d'écrire :

\begin{Shaded}
\begin{Highlighting}[]
\NormalTok{Salaires }\SpecialCharTok{|\textgreater{}} \FunctionTok{freqtable}\NormalTok{(race,salaire\_quint) }\SpecialCharTok{|\textgreater{}} \FunctionTok{rprop}\NormalTok{()}
\end{Highlighting}
\end{Shaded}

Dans ce tableau, on peut aussi ajouter les effectifs totaux des lignes
et par exemple ne pas mettre de décimales après la virgule :

\begin{Shaded}
\begin{Highlighting}[]
\NormalTok{Salaires }\SpecialCharTok{|\textgreater{}} \FunctionTok{freqtable}\NormalTok{(race,salaire\_quint) }\SpecialCharTok{|\textgreater{}} \FunctionTok{rprop}\NormalTok{(}\AttributeTok{n=}\NormalTok{T,}\AttributeTok{digit=}\DecValTok{0}\NormalTok{)}
\end{Highlighting}
\end{Shaded}

Bien sur, il est possible de réaliser un tableau des pourcentages en
colonne :

\begin{Shaded}
\begin{Highlighting}[]
\NormalTok{Salaires }\SpecialCharTok{|\textgreater{}} \FunctionTok{freqtable}\NormalTok{(race,salaire\_quint) }\SpecialCharTok{|\textgreater{}} \FunctionTok{cprop}\NormalTok{(}\AttributeTok{n=}\NormalTok{T,}\AttributeTok{digit=}\DecValTok{0}\NormalTok{)}
\end{Highlighting}
\end{Shaded}

Ou des pourcentages totaux (qu'on appelle aussi pourcentages conjoints)
:

\begin{Shaded}
\begin{Highlighting}[]
\NormalTok{Salaires }\SpecialCharTok{|\textgreater{}} \FunctionTok{freqtable}\NormalTok{(race,salaire\_quint) }\SpecialCharTok{|\textgreater{}} \FunctionTok{prop}\NormalTok{(}\AttributeTok{n=}\NormalTok{T,}\AttributeTok{digit=}\DecValTok{0}\NormalTok{)}
\end{Highlighting}
\end{Shaded}

\begin{tcolorbox}[enhanced jigsaw, toprule=.15mm, colbacktitle=quarto-callout-note-color!10!white, opacitybacktitle=0.6, opacityback=0, breakable, colback=white, coltitle=black, rightrule=.15mm, titlerule=0mm, toptitle=1mm, title={Exercice}, colframe=quarto-callout-note-color-frame, leftrule=.75mm, arc=.35mm, bottomtitle=1mm, bottomrule=.15mm, left=2mm]

\begin{enumerate}
\def\labelenumi{\arabic{enumi}.}
\item
  Étudier le lien entre le sexe et le niveau de salaire.
\item
  Réaliser un tableau dans le Viewer.
\end{enumerate}

\end{tcolorbox}

\subsection{Le tableau et son
graphique}\label{le-tableau-et-son-graphique}

ON GARDE

On peut aussi réaliser un diagramme à barres empilées ou adjacentes de
l'association entre la catégorie raciale et le niveau de salaire :

\begin{figure}[H]

{\centering \includegraphics{images/clipboard-2502740168.png}

}

\caption{Graphique à barres avec Esquisse}

\end{figure}%

ON ENLEVE

On pourrait améliorer ce graphique de deux manières :

\begin{itemize}
\item
  En inversant l'ordre des étiquettes empilées (Très élevé en haut, très
  faible en bas)
\item
  En mettant l'axe des Y en pourcentages (attention, il faut avoir
  installer le package scales)
\end{itemize}

\begin{Shaded}
\begin{Highlighting}[]
\FunctionTok{ggplot}\NormalTok{(Salaires) }\SpecialCharTok{+}
  \FunctionTok{aes}\NormalTok{(}\AttributeTok{x =}\NormalTok{ race, }\AttributeTok{fill =} \FunctionTok{fct\_rev}\NormalTok{(salaire\_quint)) }\SpecialCharTok{+}  \CommentTok{\# Inverser l\textquotesingle{}ordre d\textquotesingle{}empilement}
  \FunctionTok{geom\_bar}\NormalTok{(}\AttributeTok{position =} \StringTok{"fill"}\NormalTok{) }\SpecialCharTok{+}
  \FunctionTok{scale\_y\_continuous}\NormalTok{(}\AttributeTok{labels =}\NormalTok{ scales}\SpecialCharTok{::}\FunctionTok{percent\_format}\NormalTok{()) }\SpecialCharTok{+} \CommentTok{\# Axe des Y en \% avec package scales}
  \FunctionTok{scale\_fill\_brewer}\NormalTok{(}\AttributeTok{palette =} \StringTok{"PuRd"}\NormalTok{, }\AttributeTok{direction =} \SpecialCharTok{{-}}\DecValTok{1}\NormalTok{,}
                    \AttributeTok{guide =} \FunctionTok{guide\_legend}\NormalTok{(}\AttributeTok{reverse =} \ConstantTok{TRUE}\NormalTok{)) }\SpecialCharTok{+}  \CommentTok{\# Légende dans le bon ordre}
  \FunctionTok{labs}\NormalTok{(}
    \AttributeTok{x =} \StringTok{"Catégorie raciale"}\NormalTok{,}
    \AttributeTok{y =} \StringTok{"Pourcentage"}\NormalTok{,}
    \AttributeTok{fill =} \StringTok{"Quintile de salaire"}
\NormalTok{  ) }\SpecialCharTok{+}
  \FunctionTok{theme\_light}\NormalTok{() }\SpecialCharTok{+}
  \FunctionTok{theme}\NormalTok{(}
    \AttributeTok{legend.position =} \StringTok{"bottom"}\NormalTok{,}
    \AttributeTok{axis.title.y =} \FunctionTok{element\_text}\NormalTok{(}\AttributeTok{size =} \DecValTok{18}\NormalTok{L),}
    \AttributeTok{axis.title.x =} \FunctionTok{element\_text}\NormalTok{(}\AttributeTok{size =} \DecValTok{18}\NormalTok{L),}
    \AttributeTok{axis.text.y =} \FunctionTok{element\_text}\NormalTok{(}\AttributeTok{size =} \DecValTok{18}\NormalTok{L),}
    \AttributeTok{axis.text.x =} \FunctionTok{element\_text}\NormalTok{(}\AttributeTok{size =} \DecValTok{18}\NormalTok{L),}
    \AttributeTok{legend.text =} \FunctionTok{element\_text}\NormalTok{(}\AttributeTok{size =} \DecValTok{18}\NormalTok{L),}
    \AttributeTok{legend.title =} \FunctionTok{element\_text}\NormalTok{(}\AttributeTok{size =} \DecValTok{18}\NormalTok{L)}
\NormalTok{  )}
\end{Highlighting}
\end{Shaded}

\begin{figure}[H]

{\centering \includegraphics{images/clipboard-2683344242.png}

}

\caption{Graphique à barres empilées amélioré}

\end{figure}%

\begin{tcolorbox}[enhanced jigsaw, toprule=.15mm, colbacktitle=quarto-callout-tip-color!10!white, opacitybacktitle=0.6, opacityback=0, breakable, colback=white, coltitle=black, rightrule=.15mm, titlerule=0mm, toptitle=1mm, title=\textcolor{quarto-callout-tip-color}{\faLightbulb}\hspace{0.5em}{Bonus 1}, colframe=quarto-callout-tip-color-frame, leftrule=.75mm, arc=.35mm, bottomtitle=1mm, bottomrule=.15mm, left=2mm]

On peut vouloir plutôt réaliser un diagramme à barres adjacentes plutôt
que empilées. Dans ce cas, il est plus simple de repartir du tableau des
pourcentages en ligne :

\begin{Shaded}
\begin{Highlighting}[]
\CommentTok{\#On crée le tableau croisé avec \% en ligne}
\NormalTok{tab }\OtherTok{\textless{}{-}}\NormalTok{ Salaires }\SpecialCharTok{|\textgreater{}}
  \FunctionTok{freqtable}\NormalTok{(race, salaire\_quint) }\SpecialCharTok{|\textgreater{}}
  \FunctionTok{rprop}\NormalTok{(}\AttributeTok{total=}\NormalTok{F) }\CommentTok{\#On enlève les \% totaux}

\CommentTok{\# Il faut transformer ce tableau en format tidy pour le graphique}
\NormalTok{df\_tab }\OtherTok{\textless{}{-}} \FunctionTok{as.data.frame.matrix}\NormalTok{(tab) }\SpecialCharTok{|\textgreater{}} \CommentTok{\#transformation en une matrice de format data.frame}
  \FunctionTok{rownames\_to\_column}\NormalTok{(}\StringTok{"race"}\NormalTok{) }\SpecialCharTok{|\textgreater{}} \CommentTok{\#Les lignes étaient des "row.names" auxquelles on assigne un nom de colonne}
  \FunctionTok{pivot\_longer}\NormalTok{(}
    \AttributeTok{cols =} \SpecialCharTok{{-}}\NormalTok{race,}
    \AttributeTok{names\_to =} \StringTok{"salaire\_quint"}\NormalTok{,}
    \AttributeTok{values\_to =} \StringTok{"pct"}
\NormalTok{  ) }\CommentTok{\#On transforme le tableau de telle sorte que tous les \% soient dans la même colonne et les modalités de salaire également. }
\NormalTok{df\_tab}\SpecialCharTok{$}\NormalTok{salaire\_quint }\OtherTok{\textless{}{-}}\NormalTok{ df\_tab}\SpecialCharTok{$}\NormalTok{salaire\_quint }\SpecialCharTok{|\textgreater{}}
  \FunctionTok{fct\_relevel}\NormalTok{(}
    \StringTok{"Très faible"}\NormalTok{,}\StringTok{"Faible"}\NormalTok{,}\StringTok{"Moyen"}\NormalTok{,}\StringTok{"Elevé"}\NormalTok{,}\StringTok{"Très élevé"}
\NormalTok{  )}

\FunctionTok{ggplot}\NormalTok{(df\_tab, }\FunctionTok{aes}\NormalTok{(}\AttributeTok{x =}\NormalTok{ salaire\_quint, }\AttributeTok{y =}\NormalTok{ pct }\SpecialCharTok{/} \DecValTok{100}\NormalTok{, }\AttributeTok{fill =}\NormalTok{ race)) }\SpecialCharTok{+}
  \FunctionTok{geom\_bar}\NormalTok{(}\AttributeTok{stat =} \StringTok{"identity"}\NormalTok{,}\AttributeTok{position =} \FunctionTok{position\_dodge2}\NormalTok{(}\AttributeTok{width =} \FloatTok{0.9}\NormalTok{,}\AttributeTok{preserve=}\StringTok{"single"}\NormalTok{))}\SpecialCharTok{+}
  \FunctionTok{geom\_text}\NormalTok{(}\FunctionTok{aes}\NormalTok{(}\AttributeTok{label =} \FunctionTok{paste0}\NormalTok{(}\FunctionTok{round}\NormalTok{(pct,}\DecValTok{0}\NormalTok{), }\StringTok{"\%"}\NormalTok{)),}
             \AttributeTok{position =} \FunctionTok{position\_dodge2}\NormalTok{(}\AttributeTok{width =} \FloatTok{0.9}\NormalTok{,}\AttributeTok{preserve=}\StringTok{"single"}\NormalTok{),}
             \AttributeTok{vjust=}\SpecialCharTok{{-}}\NormalTok{.}\DecValTok{3}\NormalTok{,}
             \AttributeTok{size =} \DecValTok{5}\NormalTok{) }\SpecialCharTok{+}
  \FunctionTok{scale\_y\_continuous}\NormalTok{(}\AttributeTok{lim=}\FunctionTok{c}\NormalTok{(}\DecValTok{0}\NormalTok{,.}\DecValTok{45}\NormalTok{),}\AttributeTok{labels =}\NormalTok{ scales}\SpecialCharTok{::}\FunctionTok{percent\_format}\NormalTok{(}\AttributeTok{accuracy =} \DecValTok{1}\NormalTok{)) }\SpecialCharTok{+}
  \FunctionTok{scale\_fill\_brewer}\NormalTok{(}\AttributeTok{palette =} \StringTok{"Dark2"}\NormalTok{, }\AttributeTok{direction =} \SpecialCharTok{{-}}\DecValTok{1}\NormalTok{,}
                    \AttributeTok{guide =} \FunctionTok{guide\_legend}\NormalTok{(}\AttributeTok{reverse =}\NormalTok{ T)) }\SpecialCharTok{+}
  \FunctionTok{labs}\NormalTok{(}
    \AttributeTok{x =} \StringTok{"Quintile de salaire"}\NormalTok{,}
    \AttributeTok{y =} \StringTok{"Pourcentage"}\NormalTok{,}
    \AttributeTok{fill =} \StringTok{"Catégorie raciale"}
\NormalTok{  ) }\SpecialCharTok{+}
  \FunctionTok{theme\_light}\NormalTok{() }\SpecialCharTok{+}
  \FunctionTok{theme}\NormalTok{(}
    \AttributeTok{legend.position =} \StringTok{"bottom"}\NormalTok{,}
    \AttributeTok{axis.title.y =} \FunctionTok{element\_text}\NormalTok{(}\AttributeTok{size =} \DecValTok{18}\NormalTok{L),}
    \AttributeTok{axis.title.x =} \FunctionTok{element\_text}\NormalTok{(}\AttributeTok{size =} \DecValTok{18}\NormalTok{L),}
    \AttributeTok{axis.text.y =} \FunctionTok{element\_text}\NormalTok{(}\AttributeTok{size =} \DecValTok{18}\NormalTok{L),}
    \AttributeTok{axis.text.x =} \FunctionTok{element\_text}\NormalTok{(}\AttributeTok{size =} \DecValTok{18}\NormalTok{L),}
    \AttributeTok{legend.text =} \FunctionTok{element\_text}\NormalTok{(}\AttributeTok{size =} \DecValTok{18}\NormalTok{L),}
    \AttributeTok{legend.title =} \FunctionTok{element\_text}\NormalTok{(}\AttributeTok{size =} \DecValTok{18}\NormalTok{L)}
\NormalTok{  )}
\end{Highlighting}
\end{Shaded}

\begin{figure}[H]

{\centering \includegraphics{images/clipboard-1796684723.png}

}

\caption{Diagramme à barres adjacentes}

\end{figure}%

\end{tcolorbox}

\begin{tcolorbox}[enhanced jigsaw, toprule=.15mm, colbacktitle=quarto-callout-tip-color!10!white, opacitybacktitle=0.6, opacityback=0, breakable, colback=white, coltitle=black, rightrule=.15mm, titlerule=0mm, toptitle=1mm, title=\textcolor{quarto-callout-tip-color}{\faLightbulb}\hspace{0.5em}{Bonus 2}, colframe=quarto-callout-tip-color-frame, leftrule=.75mm, arc=.35mm, bottomtitle=1mm, bottomrule=.15mm, left=2mm]

On pourrait aussi vouloir créer un diagramme à barres avec les
étiquettes de valeurs de pourcentage sur le graphique. Dans ce cas, il
est plus simple de repartir du tableau des pourcentages en ligne :

\begin{Shaded}
\begin{Highlighting}[]
\CommentTok{\#On crée le tableau croisé avec \% en ligne}
\NormalTok{tab }\OtherTok{\textless{}{-}}\NormalTok{ Salaires }\SpecialCharTok{|\textgreater{}}
  \FunctionTok{freqtable}\NormalTok{(race, salaire\_quint) }\SpecialCharTok{|\textgreater{}}
  \FunctionTok{rprop}\NormalTok{()}

\CommentTok{\# Il faut transformer ce tableau en format tidy pour le graphique}
\NormalTok{df\_tab }\OtherTok{\textless{}{-}} \FunctionTok{as.data.frame.matrix}\NormalTok{(tab) }\SpecialCharTok{|\textgreater{}} \CommentTok{\#transformation en une matrice de format data.frame}
  \FunctionTok{rownames\_to\_column}\NormalTok{(}\StringTok{"race"}\NormalTok{) }\SpecialCharTok{|\textgreater{}} \CommentTok{\#Les lignes étaient des "row.names" auxquelles on assigne un nom de colonne}
  \FunctionTok{select}\NormalTok{(}\SpecialCharTok{{-}}\NormalTok{Total) }\SpecialCharTok{|\textgreater{}} \CommentTok{\#On enlève les lignes de total}
  \FunctionTok{pivot\_longer}\NormalTok{(}
    \AttributeTok{cols =} \SpecialCharTok{{-}}\NormalTok{race,}
    \AttributeTok{names\_to =} \StringTok{"salaire\_quint"}\NormalTok{,}
    \AttributeTok{values\_to =} \StringTok{"pct"}
\NormalTok{  ) }\CommentTok{\#On transforme le tableau de telle sorte que tous les \% soient dans la même colonne et les modalités de salaire également. }
\NormalTok{df\_tab}\SpecialCharTok{$}\NormalTok{salaire\_quint }\OtherTok{\textless{}{-}}\NormalTok{ df\_tab}\SpecialCharTok{$}\NormalTok{salaire\_quint }\SpecialCharTok{|\textgreater{}}
  \FunctionTok{fct\_relevel}\NormalTok{(}
    \StringTok{"Très faible"}\NormalTok{,}\StringTok{"Faible"}\NormalTok{,}\StringTok{"Moyen"}\NormalTok{,}\StringTok{"Elevé"}\NormalTok{,}\StringTok{"Très élevé"}
\NormalTok{  )}

\CommentTok{\# Graphique}
\FunctionTok{ggplot}\NormalTok{(df\_tab, }\FunctionTok{aes}\NormalTok{(}\AttributeTok{x =}\NormalTok{ race, }\AttributeTok{y =}\NormalTok{ pct }\SpecialCharTok{/} \DecValTok{100}\NormalTok{, }\AttributeTok{fill =} \FunctionTok{fct\_rev}\NormalTok{(salaire\_quint))) }\SpecialCharTok{+}
  \FunctionTok{geom\_bar}\NormalTok{(}\AttributeTok{stat =} \StringTok{"identity"}\NormalTok{) }\SpecialCharTok{+}
  \FunctionTok{geom\_label}\NormalTok{(}\FunctionTok{aes}\NormalTok{(}\AttributeTok{label =} \FunctionTok{paste0}\NormalTok{(}\FunctionTok{round}\NormalTok{(pct,}\DecValTok{0}\NormalTok{), }\StringTok{"\%"}\NormalTok{)),}
            \AttributeTok{position =} \FunctionTok{position\_stack}\NormalTok{(}\AttributeTok{vjust =} \FloatTok{0.5}\NormalTok{),}
            \AttributeTok{size =} \DecValTok{5}\NormalTok{) }\SpecialCharTok{+}
  \FunctionTok{scale\_y\_continuous}\NormalTok{(}\AttributeTok{labels =}\NormalTok{ scales}\SpecialCharTok{::}\FunctionTok{percent\_format}\NormalTok{(}\AttributeTok{accuracy =} \DecValTok{1}\NormalTok{)) }\SpecialCharTok{+}
  \FunctionTok{scale\_fill\_brewer}\NormalTok{(}\AttributeTok{palette =} \StringTok{"PuRd"}\NormalTok{, }\AttributeTok{direction =} \SpecialCharTok{{-}}\DecValTok{1}\NormalTok{,}
                    \AttributeTok{guide =} \FunctionTok{guide\_legend}\NormalTok{(}\AttributeTok{reverse =}\NormalTok{ T)) }\SpecialCharTok{+}
  \FunctionTok{labs}\NormalTok{(}
    \AttributeTok{x =} \StringTok{"Catégorie raciale"}\NormalTok{,}
    \AttributeTok{y =} \StringTok{"Pourcentage"}\NormalTok{,}
    \AttributeTok{fill =} \StringTok{"Quintile de salaire"}
\NormalTok{  ) }\SpecialCharTok{+}
  \FunctionTok{theme\_light}\NormalTok{() }\SpecialCharTok{+}
  \FunctionTok{theme}\NormalTok{(}
    \AttributeTok{legend.position =} \StringTok{"bottom"}\NormalTok{,}
    \AttributeTok{axis.title.y =} \FunctionTok{element\_text}\NormalTok{(}\AttributeTok{size =} \DecValTok{18}\NormalTok{L),}
    \AttributeTok{axis.title.x =} \FunctionTok{element\_text}\NormalTok{(}\AttributeTok{size =} \DecValTok{18}\NormalTok{L),}
    \AttributeTok{axis.text.y =} \FunctionTok{element\_text}\NormalTok{(}\AttributeTok{size =} \DecValTok{18}\NormalTok{L),}
    \AttributeTok{axis.text.x =} \FunctionTok{element\_text}\NormalTok{(}\AttributeTok{size =} \DecValTok{18}\NormalTok{L),}
    \AttributeTok{legend.text =} \FunctionTok{element\_text}\NormalTok{(}\AttributeTok{size =} \DecValTok{18}\NormalTok{L),}
    \AttributeTok{legend.title =} \FunctionTok{element\_text}\NormalTok{(}\AttributeTok{size =} \DecValTok{18}\NormalTok{L)}
\NormalTok{  )}
\end{Highlighting}
\end{Shaded}

\begin{figure}[H]

{\centering \includegraphics{images/clipboard-3779074093.png}

}

\caption{Diagramme à barres empilées avec étiquettes}

\end{figure}%

\end{tcolorbox}

\begin{tcolorbox}[enhanced jigsaw, toprule=.15mm, colbacktitle=quarto-callout-note-color!10!white, opacitybacktitle=0.6, opacityback=0, breakable, colback=white, coltitle=black, rightrule=.15mm, titlerule=0mm, toptitle=1mm, title={Exercice}, colframe=quarto-callout-note-color-frame, leftrule=.75mm, arc=.35mm, bottomtitle=1mm, bottomrule=.15mm, left=2mm]

Étudier le lien entre sexe et niveau de salaire à l'aide d'un diagramme
à barres empilées ou adjacentes.

\end{tcolorbox}

\subsection{La force de l'association}\label{la-force-de-lassociation}

ON ENLEVE

Pour étudier l'intensité de l'association entre deux variables
qualitatives, l'indice le plus évident est le V de Cramer qui prend une
valeur entre 0 (pas d'association) et 1 (association parfaite).

À partir de 0,2 ou 0,3, on pourra considérer qu'il y a une association
notable.

On le calcule ainsi :

\begin{Shaded}
\begin{Highlighting}[]
\NormalTok{tab }\OtherTok{\textless{}{-}}\NormalTok{ Salaires }\SpecialCharTok{|\textgreater{}}
  \FunctionTok{freqtable}\NormalTok{(race, salaire\_quint)}
\FunctionTok{cramer.v}\NormalTok{(tab)}
\end{Highlighting}
\end{Shaded}

\begin{tcolorbox}[enhanced jigsaw, toprule=.15mm, colbacktitle=quarto-callout-note-color!10!white, opacitybacktitle=0.6, opacityback=0, breakable, colback=white, coltitle=black, rightrule=.15mm, titlerule=0mm, toptitle=1mm, title={Exercice}, colframe=quarto-callout-note-color-frame, leftrule=.75mm, arc=.35mm, bottomtitle=1mm, bottomrule=.15mm, left=2mm]

L'intensité de l'association est-elle plus forte pour entre le salaire
et le sexe ou le salaire et la catégorie raciale ?

\end{tcolorbox}

\subsection{Un mot sur le test du
khi-deux}\label{un-mot-sur-le-test-du-khi-deux}

ON DEPLACE PLUS LOIN

Difficile de ne pas évoquer pour finir le test du chi-deux, sans
toutefois trop s'y attarder. Précisons que Julien Barnier a écrit un
\href{https://raw.githubusercontent.com/juba/archive_doc_khi2/master/khi2.pdf}{récapitalutif
très exhaustif} sur ce qu'est et n'est pas le test du khi-deux (ou
chi-deux ou \chi-deux ou chi-squared\ldots).

À quoi sert le test du khi-deux :

\begin{itemize}
\item
  Déterminer la probabilité que les lignes et les colonnes du tableau
  croisé sont indépendantes
\item
  Évaluer si la répartition des effectifs dans une table de contingence
  est significativement différente de la table calculée sous l'hypothèse
  d'indépendance des deux variables croisées
\end{itemize}

\emph{La distribution du salaire est-elle indépendante de celle de la
catégorie raciale des individus ? Y-a-t-il une association entre race et
salaire ?}

\begin{tcolorbox}[enhanced jigsaw, toprule=.15mm, colbacktitle=quarto-callout-note-color!10!white, opacitybacktitle=0.6, opacityback=0, breakable, colback=white, coltitle=black, rightrule=.15mm, titlerule=0mm, toptitle=1mm, title=\textcolor{quarto-callout-note-color}{\faInfo}\hspace{0.5em}{Explications sur le test du khi-deux}, colframe=quarto-callout-note-color-frame, leftrule=.75mm, arc=.35mm, bottomtitle=1mm, bottomrule=.15mm, left=2mm]

On va poser un test statistique, avec une hypothèse H0, ou hypothèse
nulle qui est que :

\emph{La distribution des quintiles de salaire est indépendante de la
catégorie raciale.}

On cherche à savoir à quoi ressemblerait notre tableau croisé si les
deux variables étaient effectivement indépendantes l'une de l'autre et
quelle est la probabilité (la p-valeur) pour que les deux variables
soient effectivement indépendantes l'une de l'autre, modulo nos
fluctuations d'échantillonnage.

En pratique, les variables sont indépendantes si :

\begin{itemize}
\item
  Les pourcentages lignes du tableau croisé sont les mêmes pour toutes
  les lignes
\item
  Les pourcentages colonnes du tableau croisé sont les mêmes pour toutes
  les colonnes
\end{itemize}

Dans notre exemple, lignes et colonnes ne semblent pas très
indépendantes.

Mais dans un échantillon issu d'une enquête, il est rare que les
variables croisées soient parfaitement indépendantes car les données du
tableau sont dépendantes de l'échantillon interrogé et tout échantillon
est soumis à des biais (qui, si l'échantillon a été construit de manière
aléatoire, sont dus \emph{au hasard,} donc des \emph{fluctuations
d'échantillonage}).

Le test du khi-deux permet de savoir à partir de quel seuil on peut
estimer que les variations observées par rapport à la situation
d'indépendante sont dues au hasard et à partir de quand elles sont dues
à un lien entre les variables.

À partir du tableau des effectifs, on peut calculer les effectifs
théoriques, \emph{si les deux variables étaient indépendantes}, grâce
aux marges des lignes et des colonnes :

\emph{Effectif théorique d'une cellule = (total ligne x total colonne) /
total global}

Ce qu'on peut calculer manuellement à partir des marges (les totaux)
qu'on peut obtenir par exemple en faisant ainsi :

\begin{Shaded}
\begin{Highlighting}[]
\NormalTok{Salaires }\SpecialCharTok{|\textgreater{}}
  \FunctionTok{freqtable}\NormalTok{(race, salaire\_quint) }\SpecialCharTok{|\textgreater{}}
  \FunctionTok{prop}\NormalTok{(}\AttributeTok{n=}\NormalTok{T)}
\end{Highlighting}
\end{Shaded}

L'effectif théorique de la cellule Blanc x Très faible est donc 90,6 car
:

\begin{Shaded}
\begin{Highlighting}[]
\DecValTok{110}\SpecialCharTok{*}\DecValTok{440}\SpecialCharTok{/}\DecValTok{534}
\end{Highlighting}
\end{Shaded}

Heureusement R dispose de la fonction chisq.test() qui permet notamment
d'afficher les effectifs théoriques :

\begin{Shaded}
\begin{Highlighting}[]
\NormalTok{tab}\OtherTok{\textless{}{-}}\NormalTok{Salaires }\SpecialCharTok{|\textgreater{}}
  \FunctionTok{freqtable}\NormalTok{(race, salaire\_quint)}
\NormalTok{chi}\OtherTok{\textless{}{-}}\FunctionTok{chisq.test}\NormalTok{(tab)}
\NormalTok{chi}\SpecialCharTok{$}\NormalTok{expected}
\end{Highlighting}
\end{Shaded}

Pour mémoire les effectifs observés sont :

\begin{Shaded}
\begin{Highlighting}[]
\NormalTok{chi}\SpecialCharTok{$}\NormalTok{observed}
\end{Highlighting}
\end{Shaded}

Alors à quel point les effectifs théoriques divergent des effectifs
observés ?

Pour ce faire, on calcule des ``khi-deux partiels'' définis comme la
distance standardisée entre les effectifs théoriques et observés :

\begin{figure}[H]

{\centering \includegraphics{images/clipboard-1078637929.png}

}

\caption{Définition des khi-deux partiels}

\end{figure}%

Ces khi-deux partiels peuvent être calculés comme ceci :

\begin{Shaded}
\begin{Highlighting}[]
\NormalTok{(chi}\SpecialCharTok{$}\NormalTok{expected}\SpecialCharTok{{-}}\NormalTok{chi}\SpecialCharTok{$}\NormalTok{observed)}\SpecialCharTok{\^{}}\DecValTok{2}\SpecialCharTok{/}\NormalTok{chi}\SpecialCharTok{$}\NormalTok{expected}
\end{Highlighting}
\end{Shaded}

\begin{figure}[H]

{\centering \includegraphics{images/clipboard-2425509646.png}

}

\caption{Khi-deux partiels}

\end{figure}%

Ici, on voit déjà que les écarts à l'indépendance sont les plus élevés
pour la cellule HispaniquexTrès faible, ce que nous avions repéré avec
une surreprésentation de cette catégorie raciale parmi ceux qui gagnent
de faibles salaires.

On calcule ensuite la valeur du khi-deux du tableau, qui correspond à la
somme des khi-deux partiels :

\begin{Shaded}
\begin{Highlighting}[]
\FunctionTok{sum}\NormalTok{((chi}\SpecialCharTok{$}\NormalTok{expected}\SpecialCharTok{{-}}\NormalTok{chi}\SpecialCharTok{$}\NormalTok{observed)}\SpecialCharTok{\^{}}\DecValTok{2}\SpecialCharTok{/}\NormalTok{chi}\SpecialCharTok{$}\NormalTok{expected)}
\end{Highlighting}
\end{Shaded}

Qu'on obtient aussi ici :

\begin{Shaded}
\begin{Highlighting}[]
\NormalTok{chi}\SpecialCharTok{$}\NormalTok{statistic}
\end{Highlighting}
\end{Shaded}

Est-ce que cette valeur est faible ou élevée ?

On va pouvoir comparer cette statistique à la ``loi du khi-deux'', une
distribution statistique qui nous donne les valeurs théoriques du
khi-deux d'un tableau sous condition d'indépendance.

Cette loi dépend d'un paramètre : le nombre de degrés de libertés. Quel
est le nombre de degrés de libertés ici ?

Il dépend grosso modo de la taille du tableau :

\emph{(Nombre de lignes - 1) x (Nombre de colonnes - 1)}

Ici, nous avons 3 lignes et 5 colonnes, donc le degré de liberté (ddl)
est égal à 8.

On le vérifie ici :

\begin{Shaded}
\begin{Highlighting}[]
\NormalTok{chi}\SpecialCharTok{$}\NormalTok{parameter}
\end{Highlighting}
\end{Shaded}

\begin{figure}[H]

{\centering \includegraphics{images/clipboard-4241100990.png}

}

\caption{Loi du khi-deux suivant le degré de libertés (k)}

\end{figure}%

On va enfin calculer une probabilité (une p-valeur), qui correspond à
l'aire sous la courbe théorique à droite de la valeur du khi-deux
observée, qui ici vaut 0,397, comme indiqué sur la figure.

En effet, rappelons-nous, la p-valeur c'est la probabilité que le khi-2
théorique soit supérieur ou égal au khi-2 observé sous condition
d'indépendance.

Ici, on voit que si les deux variables sont indépendantes, il y a une
probabilité de presque 40~\% pour que par hasard on obtienne une valeur
du khi-2 au moins aussi élevée que celle qu'on a observé (8,374).

Autrement dit, notre valeur du khi-deux est assez plausible / compatible
avec le fait que les deux variables soient indépendantes !

Cette p-valeur nous empêche de rejeter l'hypothèse d'indépendance entre
catégorie racial et salaire.

\emph{En sciences sociales, on retient généralement le seuil d'une
p-valeur de 0,05 en dessous de laquelle on estime qu'on peut rejeter
l'hypothèse d'indépendance.}

Pour ce faire, il aurait fallu ici obtenir un khi-2 observé au moins
égal à 15,507 (on en est loin !).

\begin{figure}[H]

{\centering \includegraphics{images/clipboard-3821422757.png}

}

\caption{Distribution du khi-2 et p-valeur}

\end{figure}%

Alors que conclure ? On ne rejette pas H0, l'hypothèse d'indépendance.

Mais pouvons-nous affirmer que le salaire ne dépend pas de la catégorie
raciale ? Eh bien, pas vraiment non plus. En gros, on n'a pas rejeté
l'idée que les deux variables sont indépendantes, et les variations
observées peuvent être dues à des fluctuations d'échantillonnage\ldots{}
ou pas.

Et si la p-valeur avait été inférieure à 0,05 ? On aurait pu rejeter H0
l'hypothèse d'indépendance et on aurait considéré qu'un lien existe
entre les deux variables.

On peut vérifier la p-valeur calculée comme ceci :

\begin{Shaded}
\begin{Highlighting}[]
\NormalTok{chi}\SpecialCharTok{$}\NormalTok{p.value}
\end{Highlighting}
\end{Shaded}

\end{tcolorbox}

On pose l'hypothèse nulle (H0) que les deux variables sont
indépendantes. On fixe un seuil de significativité pour la p-valeur égal
à 0,05, en dessous de laquelle on rejette l'hypothèse d'indépendance et
à ce moment là on considère qu'un lien existe entre les variables :

\begin{Shaded}
\begin{Highlighting}[]
\NormalTok{tab}\OtherTok{\textless{}{-}}\NormalTok{Salaires }\SpecialCharTok{|\textgreater{}}
  \FunctionTok{freqtable}\NormalTok{(race, salaire\_quint)}
\NormalTok{chi}\OtherTok{\textless{}{-}}\FunctionTok{chisq.test}\NormalTok{(tab)}
\NormalTok{chi}
\end{Highlighting}
\end{Shaded}

La p-valeur est supérieure à 0,05, donc on ne peut pas rejeter H0 et on
ne peut pas conclure qu'un lien existe entre les deux variables.

\begin{tcolorbox}[enhanced jigsaw, toprule=.15mm, colbacktitle=quarto-callout-note-color!10!white, opacitybacktitle=0.6, opacityback=0, breakable, colback=white, coltitle=black, rightrule=.15mm, titlerule=0mm, toptitle=1mm, title=\textcolor{quarto-callout-note-color}{\faInfo}\hspace{0.5em}{Exercice}, colframe=quarto-callout-note-color-frame, leftrule=.75mm, arc=.35mm, bottomtitle=1mm, bottomrule=.15mm, left=2mm]

Tester l'hypothèse d'un lien entre le sexe et les quintiles de salaire.

\end{tcolorbox}

ON MET ICI le RECODAGE DES VARIABLES QUALIS

\section{Exercices}\label{exercices}

ON ALLEGE

\begin{tcolorbox}[enhanced jigsaw, toprule=.15mm, colbacktitle=quarto-callout-note-color!10!white, opacitybacktitle=0.6, opacityback=0, breakable, colback=white, coltitle=black, rightrule=.15mm, titlerule=0mm, toptitle=1mm, title={Exercice}, colframe=quarto-callout-note-color-frame, leftrule=.75mm, arc=.35mm, bottomtitle=1mm, bottomrule=.15mm, left=2mm]

\begin{enumerate}
\def\labelenumi{\arabic{enumi}.}
\item
  Créer une variable haut\_revenu qui vaut ``oui'' si un individu
  appartient aux 10 \% les mieux rémunérés (salaire ≥ 9e décile) et
  ``non'' sinon. Calculer la proportion d'individus ayant un haut revenu
  pour le genre et la race, calculer les V de Cramer et réaliser des
  tests du khi-deux. Réaliser des diagrammes à barres. Qui sont les plus
  présents en haut de l'échelle salariale ?
\item
  Recoder la variable âge en quatre groupes :
\end{enumerate}

\begin{itemize}
\item
  Moins de 30 ans
\item
  30-44 ans
\item
  45-59 ans
\item
  60 ans et plus
\end{itemize}

\begin{enumerate}
\def\labelenumi{\arabic{enumi}.}
\setcounter{enumi}{2}
\item
  Pour chaque groupe d'âge, calculer la médiane et les quartiles du
  salaire et de l'expérience. Représenter ces distributions avec des
  boxplots. Observe-t-on un effet d'âge linéaire ?
\item
  Décrire à l'aide d'un tableau croisé la distribution de l'âge en
  fonction des hauts revenus. Calculer le V de Cramer et réaliser un
  test du khi-deux. Réaliser un diagramme à barres. Qui sont les plus
  présents en haut de l'échelle salariale ?
\end{enumerate}

\end{tcolorbox}

\part{\textbf{Topics of interest}}

\chapter{}\label{section}

ON MET LES DIFFERENTS TRAITEMENTS R.

Les tableaux finaux.

On met des notes de lecture.

\chapter{}\label{section-1}

To be written.

\part{\textbf{To move further}}

\chapter{Organisation du code et manipulations
avancées}\label{organisation-du-code-et-manipulations-avancuxe9es}

A TRADUIRE

Ce chapitre vise surtout à synthétiser quelques trucs et astuces pour se
faciliter la vie dans l'utilisation de R.

\section{La notion de ``projet'' dans
RStudio}\label{la-notion-de-projet-dans-rstudio}

Tout au long de ces chapitres, nous avons créé différents scripts R
rassemblés dans un dossier ``Formation R''. Une bonne pratique est de
créer un ``projet'' pointant vers le dossier des scripts qui ont une
cohérence ensemble (un projet de recherche par exemple), les données
afférentes, etc.

Ce dossier projet peut être créé en cliquant en haut à droite sur
``Project: (None)'', puis ``New Project'' et ici ``Existing Directory''
(on sélectionnera alors le dossier Formation R.

Ce dossier peut contenir plusieurs sous-dossiers :

\begin{itemize}
\item
  Script
\item
  Data
\item
  Output (là où éventuellement on rassemble ses graphiques et autres
  jolis tableaux)
\end{itemize}

L'intérêt de fonctionner par projet est que RStudio va ensuite
automatiquement pointer vers le dossier racine (Formation R) quand le
projet est ouvert et on accèdera plus facilement à ses fichiers dans le
Files et en écrivant le chemin d'accès dans ses scripts.

\section{Organiser son code et ses
fichiers}\label{organiser-son-code-et-ses-fichiers}

Il est assez courant de commencer un projet et d'y revenir des mois
voire des années plus tard.

Que faire quand on a tout oublié ?

Si on a bien structuré son code, ce qu'on peut faire en ajoutant des
sections, dans son script, on s'y retrouve plus facilement :

\begin{Shaded}
\begin{Highlighting}[]
\CommentTok{\# Ceci marque une section {-}{-}{-}{-}{-}{-}{-}{-}{-}{-}{-}{-}{-}{-}{-}{-}{-}{-}{-}{-}{-}{-}{-}{-}{-}{-}{-}{-}{-}{-}}
\end{Highlighting}
\end{Shaded}

On peut aussi commenter allégrement ses différentes opérations pour s'y
retrouver :

\begin{Shaded}
\begin{Highlighting}[]
\CommentTok{\#Le dièse permet d\textquotesingle{}initier un commentaire}
\CommentTok{\# Quand on doit faire un long commentaire,}
\CommentTok{\# on peut écrire sans dièse, puis surligner le tout,}
\CommentTok{\# et taper Cmd (ou Ctrl)+Maj+C et le bloc sera mis en}
\CommentTok{\# commentaire !}
\end{Highlighting}
\end{Shaded}

Il est aussi assez courant de créer différents scripts pour différents
types d'opérations (mais ça dépend de ses habitudes), par exemple :

\begin{itemize}
\item
  Un script ou des script(s) où je réalise tous les recodages
  nécessaires à l'analyse et j'enregistre ma base avec les nouvelles
  variables (en format R, rds plutôt).
\item
  Un ou des script(s) où je réalise toutes les opérations d'analyse.
\end{itemize}

On pourra si on le souhaite créer différents fichiers suivant
l'avancement du projet et indiquer la date dans le titre. Ainsi, le
fichier de recodage pourra s'appeler ``20250602Recodages.R'' (la date
est mise dans le format AAAAMMJJ pour être sur que les fichiers soient
présentés du plus ancien au plus récent dans le dossier où ils sont
stockés.

\section{Obtenir de l'aide}\label{obtenir-de-laide}

Listons ici quelques manières d'avoir de l'aide quand le script buggue :

\begin{itemize}
\tightlist
\item
  Le premier réflexe est de chercher comment s'utilise une fonction dans
  R :
\end{itemize}

\begin{Shaded}
\begin{Highlighting}[]
\NormalTok{?mean }\CommentTok{\#le point d\textquotesingle{}interrogation ouvre la page de description de la fonction}
\end{Highlighting}
\end{Shaded}

\begin{itemize}
\item
  On peut aussi taper son problème dans son moteur de recherche, qui
  renvoie souvent vers le forum Stackoverflow (mais pas que) où le
  problème auquel on fait face a été discuté (généralement, en anglais)
\item
  On peut consulter les guides de formation à R mentionnés sur la page
  de présentation de ce guide.
\item
  \ldots{}
\end{itemize}

\section{La jointure de différents fichiers
statistiques}\label{la-jointure-de-diffuxe9rents-fichiers-statistiques}

Nous n'avons pas ici parlé de la jointure de différents fichiers
statistiques, opération pourtant courante. Cette page de utilitR est
très complète pour mener ces opérations sans trop se prendre la tête :
\url{https://book.utilitr.org/03_Fiches_thematiques/Fiche_joindre_donnees.html}.

\section{Les étiquettes de variables et de
valeurs}\label{les-uxe9tiquettes-de-variables-et-de-valeurs}

Enfin, nous avons ici travaillé à partir de variables qui n'ont pas
d'étiquettes et dont les modalités des variables catégorielles n'ont pas
de labels. C'est pourtant possible (voire courant, si on charge un
fichier Stata dans R) et plutôt pratique :

\begin{itemize}
\item
  On a ainsi des étiquettes déjà prêtes pour les tableaux / graphiques
\item
  On va bien plus vite dans le recodage et surtout on évite les erreurs
  (recoder 1 en 3 va plus vite et comprend moins de risque d'erreur que
  recoder ``Cadres et professions scientifiques'' en ``Classes
  supérieures'').
\end{itemize}

On se reportera aux Chapitres
\href{https://larmarange.github.io/guide-R/manipulation/etiquettes-variables.html}{11}
et
\href{https://larmarange.github.io/guide-R/manipulation/etiquettes-valeurs.html}{12}
de guide-R qui introduisent à ces concepts et leur usage.

\section{Exercices}\label{exercices-1}

\begin{tcolorbox}[enhanced jigsaw, toprule=.15mm, colbacktitle=quarto-callout-note-color!10!white, opacitybacktitle=0.6, opacityback=0, breakable, colback=white, coltitle=black, rightrule=.15mm, titlerule=0mm, toptitle=1mm, title={Exercice}, colframe=quarto-callout-note-color-frame, leftrule=.75mm, arc=.35mm, bottomtitle=1mm, bottomrule=.15mm, left=2mm]

Créer un Projet pointant sur le dossier de la Formation R. Commenter et
créer des sections dans l'un de ses scripts. Pas facile ? D'où l'intérêt
de le faire au fur et à mesure :)

\end{tcolorbox}




\end{document}
